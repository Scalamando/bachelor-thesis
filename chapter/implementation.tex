\chapter{Implementierung}

In diesem Kapitel wird die Implementierung der im letzten Kapitel erarbeiteten
Konzeption für Frontend und Backend der Anwendung beschrieben.

% Um die Softwarequalität bei der gegebenen Komplexität des Frameworks
% garantieren zu können, wurden Softwaretests eingesetzt, welche die
% grundlegende Logik der verschiedenen Funktionen des Backends überprüfen
% sollten. Des Weiteren wurde ein hohes Maß an Automation angestrebt, um die
% Softwarequalität kontinuierlich zu überprüfen und die verwendete Zeit für
% repetitive Aufgaben zu minimieren.

% Starker Fokus auf Automation
%   - Github Actions
%   - Docker
%       - Compose
%       - Automatisierte Builds
%       - Watchtower

\section{Frontend}

% Vue 3
%   - vue-cli
%   - SFC
%   - MVC
%   - Style Guide / Best Practices
%       - script setup (recommended
%         https://v3.vuejs.org/api/sfc-script-setup.html)
%   - Composition API

% Typescript

% Genutzte Libraries:
%   - Axios
%   - MapboxGlJS
%   - marked (Markdown) JSDOMPurify
%   - Iconify
%   - Swiper
%   - Socket.IO

% Gliederung:
%   - Controller
%   - Service
%   - Repository

% Kamera Eigen-Implementierung

% Mapbox Eigen-Implementierung

% Modal Eigen-Implementierung

% Service Worker - Push-Benachrichtigungen

% Optimierungen:
%   - Route Splitting
%   - PWA bzw. Service Worker

\section{Backend}

% Strapi
%   - Mangelhafte Dokumentation
%   - Mangelhaftes Tooling

% Group Plugin Implementation

% Location Picker Feld Plugin Implementation

% Verschiedene APIs
%   - completion
%   - visit
%   - notification
%   - feedback

% Authentifizierung

% Socket.IO

% Dashboard

% API-Tests mit Jest

\section{Dashboard}