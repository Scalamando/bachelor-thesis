\chapter{Stand der Technik}

% Suche über Google
% Gefundene Projekte

\section{EMI-Award-App}

% Tech-Stack
%   - Hat sich größten teils bewährt
%   - Web-App mit PWA Funktionalität

% Mockups
%   - Bereits getestet
%   - Grobe Struktur beibehalten

Die Idee zu dieser Arbeit entstammt der im Rahmen eines Bachelorprojekts
entstandenen EMI-Award-App~\cite{Canzler2021}, welche bereits März 2021
eingesetzt wurde, um die verschiedenen studentischen Projekte des EMI-Awards in
der Stadt Lübeck auszustellen. Beim EMI-Award handelt es sich um eine jährliche
Veranstaltung des Studiengangs Medieninformatik, bei welcher die „die
kreativsten und gelungensten Gruppenergebnisse aus der
Erstsemester-Veranstaltung ‚Einführung in die Medieninformatik‘ (EMI)
ausgezeichnet [werden]“ \cite{UniversitatzuLubeck2021}. Die App ermöglichte
unter anderem das Einsehen, virtuelle Besuchen und Bewerten von verschiedenen
Stationen. Eine Besonderheit stellt die AR-Komponente der App dar. Nach dem
erfolgreichen Scannen des QR-Codes einer Station wurde zunächst, abhängig vom
Gerät des Nutzers, eine AR- beziehungsweise VR-Szene angezeigt, in welcher ein
Planet, symbolisch für die Station stehend, schwebte und angetippt werden
musste. Zudem konnten verschiedene Errungenschaften durch vorgegebene Aktionen
erhalten werden.

\section{Bizzabo}

\section{Eventmobi}

\section{Eventbrite}

\section{Attendify}

\section{Actionbound}