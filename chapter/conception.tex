\chapter{Konzeption}

In diesem Kapitel werden die Funktionen des Frameworks erarbeitet und
Designanforderungen festgehalten. Hierbei wurden vorhandene Evaluationen der
EMI-Award-App berücksichtigt und nach dem menschenzentrierten Gestaltungsprozess
vorgegangen. Aus den Designanforderungen wurden Konzepte für die verschiedenen
Ansichten und ihre Komponenten erstellt. Zudem wurden Konzepte für die benötigte
Infrastruktur und ihren Aufbau angefertigt.

\section{Systemarchitektur}


\section{Frontend}

% inclusive Design
%   - Mehrsprachig
%   - A11y (Aria, W3C Empfehlungen)

% Auf "Refactoring UI"-Standards geachtet

% Tailwindcss Design-System verwendet

\section{Backend}

% Komplexe Modellierung: Gruppen / Einzel

% Darstellung der Datenbank Relation

% Strukturierung der API
%   - Gruppen API
%   - Besuchs API
%   - Benachrichtigungs API
%   - Abzeichen API

\section{Implikation für Implementierung}

% Gedanken zur technischen Umsetzung der EMI-App
%   - Vor-/Nachteile Web/Native
%   - Library Entscheidungen