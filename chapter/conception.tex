\chapter{Konzeption}

Nisi et sed provident esse accusamus consequuntur praesentium qui. Eaque vel non dolores aliquam fuga voluptas quia sit. Vel ut rem et in quis quo inventore quidem. Enim quam voluptatum atque et. Consequuntur repellendus quia voluptate vel quia et suscipit soluta. Fugiat iste corporis voluptatem molestiae.

\section{Funktionalität}

Im Folgenden werden die konzipierten Funktionalitäten des Systems vorgestellt,
welche die zuvor festgelegten Anforderungen (s. \autoref{sec:analysis-anf})
erfüllen sollen. Zuerst werden die übernommenen und überarbeiteten
Funktionalitäten der EMI-Award-App dargelegt. Anschließend werden neue
Funktionalitäten vorgestellt.

\subsection{Übernommene Funktionalitäten}

Die Funktionalitäten der EMI-Award-App dienen für diese Arbeit als Grundlage. Um
die Systemarchitektur später einfacher abbilden zu können, werden die
Funktionalitäten nach Nutzungsgruppe gruppiert (s. \autoref{table:funk-old}).
Aufgrund der abweichenden Anforderungen zur EMI-Award-App müssen einige
Funktionalitäten angepasst werden. Im Folgenden werden die übernommenen und
überarbeiteten Funktionalitäten für Veranstaltende und Teilnehmende präsentiert.

\begin{table}[htpb]
    \def\arraystretch{1.25}
    \centering
    \caption{Übernommene Funktionalitäten der EMI-Award-App}
    \label{table:funk-old}
    \begin{tabular}{lll}
        \uzlhline%
        \uzlemph{ID} & \uzlemph{Titel}                            & \uzlemph{Anforderungen} \\
        \uzlhline%
        Ft-V-1       & Eintragen und Verwalten von Stationen      & \anfref{F11}            \\
        Ft-V-2       & Eintragen und Verwalten von Abzeichen      & \anfref{F12}            \\
        Ft-V-3       & Eintragen und Verwalten von Hilfseinträgen & \anfref{F13}            \\
        Ft-V-4       & Eintragen und Verwalten der Einführung     & \anfref{F14}            \\
        Ft-T-1       & Interaktive Karte mit Stationen            & \anfref{F30}            \\
        Ft-T-2       & Auflistung der Stationen                   & \anfref{F30}            \\
        Ft-T-3       & Virtuelles Besuchen mit QR-Code            &                         \\
        Ft-T-4       & Abzeichen                                  & \anfref{F60}            \\
        Ft-T-5       & Bedienungshilfe                            & \anfref{F50}            \\
        Ft-T-6       & Einleitende Slideshow                      & \anfref{F40}            \\
        \uzlhline
    \end{tabular}
\end{table}

Für die Veranstaltenden wurden einige Anpassungen vorgenommen. Das Eintragen der
Stationen (Ft-V-1), Abzeichen (Ft-V-2), Hilfseinträge (Ft-V-3) und Einführung
(Ft-V-4) muss an die Verallgemeinerung des Frameworks angepasst werden. Da der
Kontext der EMI-Award-App sehr spezifisch ist, waren die Anpassungsmöglichkeiten
auf das nötige beschränkt. Konkret wurden einige Daten in der App fest
eingebaut. Dies umschließt die Icons von Stationen und Abzeichen, die
Abschlussbedingung der einzelnen Abzeichen, sowie die Bilder der Einführung.
Dies wird durch das Auswählen eines eigenen Icons oder Bildes, sowie der
Abschlussbedingung pro Abzeichen ersetzt. Zudem werden zwei neue
Abschlussbedingungen eingeführt: eine Textabgabe und eine Bildabgabe, welche von
Veranstaltenden manuell akzeptiert oder abgelehnt werden muss.

Die Hilfseinträge (Ft-V-3) werden weitergehend überarbeitet, um eine größere
Anzahl an Einträgen übersichtlich zu ermöglichen. Hierzu werden die Einträge
nach Kategorie erstellt und in einem „Frage \& Antwort“ (FAQ) Format angegeben.
Somit wäre eine mögliche Frage „Wann endet die Veranstaltung?“, eingeordnet in
der Kategorie „Organisatorisches“.
% TODO: Markdown wichtig?

Eine weitere Überarbeitung betrifft das Eintragen und Verwalten der Einführung
(Ft-V-4), welche nun weitere Folienformate unterstützt. Das Layout der
EMI-Award-App hat ein festgesetztes Bild mit anpassbarem Text darunter
angezeigt. Stattdessen werden drei neue Formate eingeführt: Titelfolie (Bild +
Text), Bildfolie (Text + Bild) und eine ausschließliche Textfolie.

\subsection{Neue Funktionalitäten}

\begin{table}[htpb]
    \def\arraystretch{1.25}
    \centering
    \caption{Neue Funktionalitäten}
    \label{table:funk-new}
    \begin{tabular}{lll}
        \uzlhline%
        \uzlemph{ID} & \uzlemph{Titel}                    & \uzlemph{Anforderungen}    \\
        \uzlhline%
        Ft-V-5       & Benachrichtigungen an Teilnehmende & \anfref{F70}               \\
        Ft-V-6       & Feedbackanfragen                   & \anfref{F80}               \\
        Ft-V-7       & Statistiken zur Veranstaltung      & \anfref{F20}               \\
        Ft-V-8       & Zentrales Dashboard zur Verwaltung & \anfref{F10}, \anfref{F90} \\
        Ft-T-7       & Gruppen                            & \anfref{F100}              \\
        Ft-T-8       & Text- und Bildabzeichen            & \anfref{F60}               \\
        \uzlhline
    \end{tabular}
\end{table}

% - Icons im Popup, da klein und durch Veranstalter bestimmbar statt generischer
%   Planet
% - Durch Distanz + Icon an Info dazu, sehr viele Informationen -> Ausblenden
%   von Sekundär Informationen (H8)


% TODO: Station und Abzeichen Aufbau festlegen
% Aufbau Station (welche Information gehört zu einer Station?)
% Aufbau Abzeichen (welche Information gehört zu Abzeichen?)
% Aufbau Hilfe (welche Information gehört zu einem Hilfseintrag?)

Der Kern der EMI-Award-App war die Darstellung der verschiedenen
Projektstandorte, sowohl als Kartenmarker, Listeneintrag und Einzelansicht. Um
diese Funktionalität auf beliebige Veranstaltungen auszuweiten, muss zunächst
definiert werden, was eine Station an Informationen beinhaltet. Aus der
EMI-Award-App konnten die folgenden Informationen entnommen werden:
\textit{Titel, Icon, Standort, ID, Kurzbeschreibung, ausführliche Beschreibung,
    mediale Inhalte}.

Aufgrund des variablen räumlichen Kontextes ist eine Verteilung der Stationen
über Kilometer große Räume möglich. Hierdurch ist die Distanz zu den
verschiedenen Stationen eine durchaus wichtige Information für Teilnehmende
(H1). Als Konsequenz wird die Distanz in verschiedenen relevanten Stellen des
User-Interfaces angezeigt, welche in den folgenden Absätzen näher erläutert
werden.


Da Abzeichen ebenfalls selbstständig erstellt werden können
sollen, wurden auch für diese die Informationen zusammengefasst: \textit{Titel,
    Beschreibung, Icon, Erfolgsbedingung}.

\section{Interface-Design}


Einige Teile der Benutzeroberfläche beachten die zehn Usability Heuristiken
nach \textcite{Nielsen1994} nicht. Die Usability Heuristiken (s.
\autoref{table:nielsen}) werden im Verlauf mit ihren IDs referenziert.

\begin{table}[htpb]
    \def\arraystretch{1.25}
    \centering
    \caption{Die Zehn Usability Heuristiken \cite{Nielsen1994}}
    \label{table:nielsen}
    \begin{tabular}{ll}
        \uzlhline%
        \uzlemph{ID} & \uzlemph{Heuristik}                                           \\
        \uzlhline%
        H1           & Sichtbarkeit des Systemstatus                                 \\
        H2           & Übereinstimmung von System und Wirklichkeit                   \\
        H3           & Nutzerkontrolle und Freiheit                                  \\
        H4           & Beständigkeit und Standards                                   \\
        H5           & Fehlervermeidung                                              \\
        H6           & Wiedererkennung statt Erinnerung                              \\
        H7           & Flexibilität und Effizienz                                    \\
        H8           & Ästhetisches und minimalistisches Design                      \\
        H9           & Hilfestellung beim Erkennen, Bewerten und Beheben von Fehlern \\
        H10          & Hilfe und Dokumentation                                       \\
        \uzlhline
    \end{tabular}
\end{table}


\section{Frameworks}

\section{Systemarchitektur}

% Speichern von Nutzerdaten ohne konkrete Anmeldung

% Usability von interaktiven Karten
% - Alter
% - Behinderung
% - Technikaffinität

% Usability QR-Code Reader



%!%%%%%%%%%%!%
%! Frontend !%
%!%%%%%%%%%%!%
% inclusive Design
%   - Mehrsprachig
%   - A11y (Aria, W3C Empfehlungen)

% Auf "Refactoring UI"-Standards geachtet

% Tailwindcss Design-System verwendet

%!%%%%%%%%%!%
%! Backend !%
%!%%%%%%%%%!%
% Komplexe Modellierung: Gruppen / Einzel

% Darstellung der Datenbank Relation

% Strukturierung der API
%   - Gruppen API
%   - Besuchs API
%   - Benachrichtigungs API
%   - Abzeichen API

\section{Implikation für Implementierung}

% Gedanken zur technischen Umsetzung der EMI-App
%   - Vor-/Nachteile Web/Native
%   - Library Entscheidungen