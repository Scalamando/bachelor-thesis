\chapter{Konzeption}

Nisi et sed provident esse accusamus consequuntur praesentium qui. Eaque vel non dolores aliquam fuga voluptas quia sit. Vel ut rem et in quis quo inventore quidem. Enim quam voluptatum atque et. Consequuntur repellendus quia voluptate vel quia et suscipit soluta. Fugiat iste corporis voluptatem molestiae.

\section{Funktionalität}

Im Folgenden werden die konzipierten Funktionalitäten des Systems vorgestellt,
welche die zuvor festgelegten Anforderungen (s. \autoref{sec:analysis-anf})
erfüllen sollen. Zuerst werden die übernommenen und überarbeiteten
Funktionalitäten der EMI-Award-App dargelegt. Anschließend werden neue
Funktionalitäten vorgestellt.

\subsection{Übernommene Funktionalitäten}

Die Funktionalitäten der EMI-Award-App dienen für diese Arbeit als Grundlage.
Jedoch beachten einige von ihnen nicht die zehn Usability Heuristiken nach
\textcite{Nielsen1994}. Die Usability Heuristiken (s. \autoref{table:nielsen})
werden im Verlauf mit ihren IDs referenziert.

\begin{table}[htpb]
    \def\arraystretch{1.25}
    \centering
    \caption{Die Zehn Usability Heuristiken \cite{Nielsen1994}}
    \label{table:nielsen}
    \begin{tabular}{ll}
        \uzlhline%
        \uzlemph{ID} & \uzlemph{Heuristik}                                           \\
        \uzlhline%
        H1           & Sichtbarkeit des Systemstatus                                 \\
        H2           & Übereinstimmung von System und Wirklichkeit                   \\
        H3           & Nutzerkontrolle und Freiheit                                  \\
        H4           & Beständigkeit und Standards                                   \\
        H5           & Fehlervermeidung                                              \\
        H6           & Wiedererkennung statt Erinnerung                              \\
        H7           & Flexibilität und Effizienz                                    \\
        H8           & Ästhetisches und minimalistisches Design                      \\
        H9           & Hilfestellung beim Erkennen, Bewerten und Beheben von Fehlern \\
        H10          & Hilfe und Dokumentation                                       \\
        \uzlhline
    \end{tabular}
\end{table}

\begin{table}[htpb]
    \def\arraystretch{1.25}
    \centering
    \caption{Übernommene Funktionalitäten der EMI-Award-App}
    \label{table:funk-old}
    \begin{tabular}{lll}
        \uzlhline%
        \uzlemph{ID} & \uzlemph{Titel}                 \\
        \uzlhline%
        Ft-1         & Interaktive Karte mit Stationen \\
        Ft-2         & Auflistung der Stationen        \\
        Ft-3         & Virtuelles Besuchen mit QR-Code \\
        Ft-4         & Abzeichen                       \\
        Ft-5         & Bedienungshilfe                 \\
        Ft-6         & Einleitende Slideshow           \\
        \uzlhline
    \end{tabular}
\end{table}

\subsection{Neue Funktionalitäten}


% TODO: Station und Abzeichen Aufbau festlegen
% Aufbau Station (welche Information gehört zu einer Station?)
% Aufbau Abzeichen (welche Information gehört zu Abzeichen?)
% Aufbau Hilfe (welche Information gehört zu einem Hilfseintrag?)

Der Kern der EMI-Award-App war die Darstellung der verschiedenen
Projektstandorte, sowohl als Kartenmarker, Listeneintrag und Einzelansicht. Um
diese Funktionalität auf beliebige Veranstaltungen auszuweiten, muss zunächst
definiert werden, was eine Station an Informationen beinhaltet. Aus der
EMI-Award-App konnten die folgenden Informationen entnommen werden:
\textit{Titel, Icon, Standort, ID, Kurzbeschreibung, ausführliche Beschreibung,
    mediale Inhalte}.

Aufgrund des variablen räumlichen Kontextes ist eine Verteilung der Stationen
über Kilometer große Räume möglich. Hierdurch ist die Distanz zu den
verschiedenen Stationen eine durchaus wichtige Information für Teilnehmende
(H1). Als Konsequenz wird die Distanz in verschiedenen relevanten Stellen des
User-Interfaces angezeigt, welche in den folgenden Absätzen näher erläutert
werden.

Das Karten-Pop-up (s. \autoref{fig:emi-intro-map}) der EMI-Award-App mit
Kurzinformationen zu Stationen benötigt durch die hinzugekommenen Anforderungen,
sowie bestehenden Problemen, eine Überarbeitung. Aufgrund der Die empfohlene
Mindestschriftgröße beträgt 16 px für Smartphones im Abstand von 30 cm
\cite{DBSV2022}. Das Pop-up verwendet jedoch eine Schriftgröße von 12 px und ist
somit zu klein. Des Weiteren nutzen Titel und Kurzbeschreibung identische
Typografie. Der Abstand ist zudem gleich mit dem Zeilenabstand der Texte. Es
liegt somit kein visueller Unterschied zwischen Titel und Kurzbeschreibung vor.
Außerdem gibt es keinen Indikator dafür, ob die Station bereits besucht wurde.
Somit müssen Nutzer:innen sich den Besucht-Status in der Stationsliste anschauen
und merken, um auf der Karte darüber informiert zu sein (H1, H6).
% - Icons im Popup, da klein und durch Veranstalter bestimmbar statt generischer
%   Planet
% - Durch Distanz + Icon an Info dazu, sehr viele Informationen -> Ausblenden
%   von Sekundär Informationen (H8)

Da Abzeichen ebenfalls selbstständig erstellt werden können
sollen, wurden auch für diese die Informationen zusammengefasst: \textit{Titel,
    Beschreibung, Icon, Erfolgsbedingung}.

\section{Systemarchitektur}

% Speichern von Nutzerdaten ohne konkrete Anmeldung

% Usability von interaktiven Karten
% - Alter
% - Behinderung
% - Technikaffinität

% Usability QR-Code Reader

\section{Frontend}

% inclusive Design
%   - Mehrsprachig
%   - A11y (Aria, W3C Empfehlungen)

% Auf "Refactoring UI"-Standards geachtet

% Tailwindcss Design-System verwendet

\section{Backend}

% Komplexe Modellierung: Gruppen / Einzel

% Darstellung der Datenbank Relation

% Strukturierung der API
%   - Gruppen API
%   - Besuchs API
%   - Benachrichtigungs API
%   - Abzeichen API

\section{Implikation für Implementierung}

% Gedanken zur technischen Umsetzung der EMI-App
%   - Vor-/Nachteile Web/Native
%   - Library Entscheidungen