\chapter{Konzeption}

Nisi et sed provident esse accusamus consequuntur praesentium qui. Eaque vel non dolores aliquam fuga voluptas quia sit. Vel ut rem et in quis quo inventore quidem. Enim quam voluptatum atque et. Consequuntur repellendus quia voluptate vel quia et suscipit soluta. Fugiat iste corporis voluptatem molestiae.

\section{Funktionalität}

Im Folgenden werden die zuvor konzipierten Funktionalitäten des Systems
vorgestellt, welche die zuvor festgelegten Anforderungen (s.
\autoref{sec:analysis-anf}) erfüllen sollen. Zuerst werden die übernommenen und
überarbeiteten Funktionalitäten der EMI-Award-App dargelegt. Anschließend werden
neue Funktionalitäten vorgestellt.

\subsection{Übernommene Funktionalitäten}

Die Funktionalitäten der EMI-Award-App dienen für diese Arbeit als Grundlage. Um
die Systemarchitektur später einfacher abbilden zu können, werden die
Funktionalitäten nach Nutzungsgruppe gruppiert (s. \autoref{table:funk-old}).
Aufgrund der abweichenden Anforderungen zur EMI-Award-App müssen einige
Funktionalitäten angepasst werden. Im Folgenden werden die übernommenen und
überarbeiteten Funktionalitäten für Veranstaltende und Teilnehmende präsentiert.

\begin{table}[htpb]
    \def\arraystretch{1.25}
    \centering
    \caption{Übernommene Funktionalitäten der EMI-Award-App}
    \label{table:funk-old}
    \begin{tabular}{lll}
        \uzlhline%
        \uzlemph{ID} & \uzlemph{Titel}                            & \uzlemph{Anforderungen} \\
        \uzlhline%
        Ft-V-1       & Eintragen und Verwalten von Stationen      & \anfref{F11}            \\
        Ft-V-2       & Eintragen und Verwalten von Abzeichen      & \anfref{F12}            \\
        Ft-V-3       & Eintragen und Verwalten von Hilfseinträgen & \anfref{F13}            \\
        Ft-V-4       & Eintragen und Verwalten der Einführung     & \anfref{F14}            \\
        Ft-T-1       & Interaktive Karte mit Stationen            & \anfref{F30}            \\
        Ft-T-2       & Auflistung der Stationen                   & \anfref{F30}            \\
        Ft-T-3       & Virtuelles Besuchen mit QR-Code            &                         \\
        Ft-T-4       & Abzeichen                                  & \anfref{F60}            \\
        Ft-T-5       & Bedienungshilfe                            & \anfref{F50}            \\
        Ft-T-6       & Einleitende Slideshow                      & \anfref{F40}            \\
        \uzlhline
    \end{tabular}
\end{table}

Für die Veranstaltenden wurden einige Anpassungen vorgenommen. Das Eintragen der
Stationen (Ft-V-1), Abzeichen (Ft-V-2), Hilfseinträge (Ft-V-3) und Einführung
(Ft-V-4) muss an die Verallgemeinerung des Frameworks angepasst werden. Da der
Kontext der EMI-Award-App sehr spezifisch ist, waren die Anpassungsmöglichkeiten
auf das nötige beschränkt. Konkret wurden einige Daten in der App fest
eingebaut. Dies umschließt die Icons von Stationen und Abzeichen, die
Abschlussbedingung der einzelnen Abzeichen, sowie die Bilder der Einführung. All
diese Einstellungen sind nun frei anpassbar.\\
Zudem werden zwei neue Abschlussbedingungen eingeführt: eine Textabgabe und
eine Bildabgabe, welche von Veranstaltenden manuell akzeptiert oder abgelehnt
werden können.

Die Hilfseinträge (Ft-V-3) werden weitergehend überarbeitet, um eine größere
Anzahl an Einträgen übersichtlich anzuzeigen. Hierzu werden die Einträge
nach Kategorie gruppiert und in einem „Frage \& Antwort“ (FAQ) Format angegeben.
Somit wäre eine mögliche Frage „Wann endet die Veranstaltung?“, eingeordnet in
der Kategorie „Organisatorisches“.
% TODO: Markdown wichtig?

Eine weitere Überarbeitung betrifft das Eintragen und Verwalten der Einführung
(Ft-V-4), welche nun weitere Folienformate unterstützt. Das Layout der
EMI-Award-App hat ein festgesetztes Bild mit anpassbarem Text darunter
angezeigt. Stattdessen werden drei neue Formate eingeführt: Titelfolie (Bild +
Text), Bildfolie (Text + Bild) und eine ausschließliche Textfolie.

\subsection{Neue Funktionalitäten}

Um alle aufgestellten Anforderungen abzudecken, wurden zudem einige neue
Funktionalitäten konzipiert. Diese werden in \autoref{table:funk-new}
präsentiert und im Folgenden näher ausgeführt.

\begin{table}[htpb]
    \def\arraystretch{1.25}
    \centering
    \caption{Neue Funktionalitäten}
    \label{table:funk-new}
    \begin{tabular}{lll}
        \uzlhline%
        \uzlemph{ID} & \uzlemph{Titel}                    & \uzlemph{Anforderungen}   \\
        \uzlhline%
        Ft-V-5       & Benachrichtigungen an Teilnehmende & \anfref{F70}              \\
        Ft-V-6       & Feedbackanfragen                   & \anfref{F80}              \\
        Ft-V-7       & Statistiken zur Veranstaltung      & \anfref{F20}              \\
        Ft-V-8       & Zentrales Dashboard zur Verwaltung & \anfref{F10}~\anfref{F90} \\
        Ft-T-7       & Gruppen                            & \anfref{F100}             \\
        Ft-T-8       & Manuelle Abzeichen                 & \anfref{F60}              \\
        \uzlhline
    \end{tabular}
\end{table}


Die Benachrichtigungen in \textit{Ft-V-5} können von Veranstaltenden über das
Dashboard in \textit{Ft-V-8} an Teilnehmende gesendet werden. Die
Benachrichtigungen bestehen dabei aus Titel und einem Textinhalt. Alle bereits
gesendeten Benachrichtigungen können in einer Tabelle eingesehen werden. Die
Teilnehmenden erhalten beim Senden einer Benachrichtigung in der Web-App eine
Push-Benachrichtigung, welche die Eingaben der Veranstaltenden beinhaltet.
Sollten Teilnehmende zum Zeitpunkt des Absendens der Benachrichtigung die
Web-App nicht offen haben, so wird die Benachrichtigung zum nächsten Öffnen der
App präsentiert.

Die Feedbackanfragen aus \textit{Ft-V-6} geschehen in ähnlicher Form wie
\textit{Ft-V-5}. Erneut können Veranstaltende über das Dashboard aus
\textit{Ft-V-8} anfragen abschicken, welche Teilnehmende mit einem
Feedbackdialog präsentiert. Im Gegensatz zur Benachrichtigung können
Veranstaltende bei einer Feedbackanfrage einen Titel und auswählbare Gründe für
ein schlechtes Feedback angeben, um jedem Veranstaltungstypen gerecht zu werden.
Teilnehmende können aus drei möglichen Bewertungen wählen: „Gut“, „Mittelmäßig“
und „Schlecht“, welche als Emojis präsentiert werden. Bei einer Wahl von
„Mittelmäßig“ oder „Schlecht“ können Teilnehmende anschließend aus den, von
Veranstaltenden angegeben, Gründen auswählen und alternativ einen eigenen Grund
angeben. Das abgesendete Feedback wird Veranstaltenden anschließend im Dashboard
präsentiert.

Zusätzlich werden auch die Statistiken aus \textit{Ft-V-7} im
Dashboard aus \textit{Ft-V-8} angezeigt. Diese werden unterteilt in „Stationen“,
„Abzeichen“, „Teilnehmende“ und „Feedback“. Zu Stationen, Abzeichen und
Teilnehmenden werden die kumulativen, täglichen und durchschnittlichen täglichen
Besuche, neue Teilnehmende und Abzeichenabschlüsse angezeigt. Für Stationen und
Abzeichen werden zusätzlich die kumulativen Besuche und Abschlüsse pro Station
und Abzeichen angezeigt. Die kumulativen und täglichen Daten werden in Form
eines Graphen angezeigt. \\
In der Kategorie Feedback werden die insgesamt eingetroffenen Antworten, ihre
Bewertungen und die drei häufigsten angegebenen Gründe für schlechte Bewertungen
angezeigt. Zudem werden alle Antworten individuell in einer Tabelle dargestellt,
in der auch die eigenen Gründe der Teilnehmenden vorhanden sind.

Das Dashboard aus \textit{Ft-V-8} beinhaltet, neben den bereits erläuterten
Funktionen, eine Ansicht zur Bearbeitung der eingereichten Text- und
Bildabzeichen, sowie Eintragung und Verwaltung der Rahmendaten. Alle offenen
Einreichungen für Abzeichen, welche eine manuelle Bearbeitung bedürfen, werden
in einer Liste angezeigt und können ausgewählt werden, um die Einreichung zu
betrachten und zu bewerten. Die Rahmendaten der Veranstaltung lassen sich
separat davon verwalten. Sie umfassen Titel, Logo, Anfang und Ende der
Veranstaltung, sowie optionale Funktionen, ein Link zu weiteren Informationen
der Veranstaltung, das initiale Zentrum der interaktiven Karte und rechtliche
Informationen wie Impressum und Datenschutzerklärung.

Des Weiteren haben Teilnehmende die Möglichkeit Aktivitäten als Gruppen
(\textit{Ft-T-7}) abzuschließen. Hierzu gehört das Besuchen von Stationen und
das Abschließen von Abzeichen. Beim ersten Aufruf wird Teilnehmenden die
Möglichkeit gegeben eine Gruppe zu erstellen oder einer Gruppe beizutreten. Dies
geschieht über das Scannen des QR-Codes eines Gruppenmitglieds oder alternativ
mit der manuellen Eingabe des Gruppen-Codes. Zudem können Teilnehmende die
Gruppe jederzeit verlassen, eine neue erstellen oder einer anderen Beitreten.
Die Gruppen-Funktion kann von Veranstaltenden im Dashboard aus \textit{Ft-V-8}
eingestellt werden. Gruppen können erforderlich, optional oder ausgeschaltet
sein.

Abschließend stellen manuelle Abzeichen aus \textit{Ft-T-8} die letzte neue
Funktionalität dar, welche die manuelle Bearbeitung durch Veranstaltende
benötigen. Dazu gehören Text- und Bildabzeichen, sowie das manuelle Bestätigen
eines Abzeichens. Durch sie erhalten Teilnehmende die Möglichkeit kleine Texte
oder Bilder bei entsprechenden Abzeichen abzugeben. Nach der Abgabe können
Veranstaltende, wie in der Dashboard-Funktionalität beschrieben, die Einreichung
akzeptieren oder ablehnen. Sollte die Einreichung abgelehnt werden, können
Teilnehmende erneut eine Abgabe einreichen. Andernfalls wird das Abzeichen als
abgeschlossen markiert. \\
Insgesamt gibt es somit nun folgende Abschlussbedingungen: \textit{Anzahl
    besuchter Stationen}, \textit{Abschließen aller (anderen) Abzeichen},
\textit{Textabgabe}, \textit{Bildabgabe} und \textit{Manuelles akzeptieren}

\section{Interface-Design}

In diesem Abschnitt wird die Benutzeroberfläche aus Sicht der Teilnehmenden und
Veranstaltenden vorgestellt. Die Grundlage der Benutzeroberfläche bildet der
Entwurf der EMI-Award-App, welcher im Rahmen des dazugehörigen Bachelorprojekts
bereits evaluiert wurde. Es wurden einige Anpassungen vorgenommen, um die
Usability Heuristiken nach \textcite{Nielsen1994} zu beachten. Diese
werden im Verlauf mit ihren IDs referenziert (s. \autoref{table:nielsen}).

\begin{table}[htpb]
    \def\arraystretch{1.25}
    \centering
    \caption{Die Zehn Usability Heuristiken \cite{Nielsen1994}}
    \label{table:nielsen}
    \begin{tabular}{ll}
        \uzlhline%
        \uzlemph{ID} & \uzlemph{Heuristik}                                           \\
        \uzlhline%
        H1           & Sichtbarkeit des Systemstatus                                 \\
        H2           & Übereinstimmung von System und Wirklichkeit                   \\
        H3           & Nutzerkontrolle und Freiheit                                  \\
        H4           & Beständigkeit und Standards                                   \\
        H5           & Fehlervermeidung                                              \\
        H6           & Wiedererkennung statt Erinnerung                              \\
        H7           & Flexibilität und Effizienz                                    \\
        H8           & Ästhetisches und minimalistisches Design                      \\
        H9           & Hilfestellung beim Erkennen, Bewerten und Beheben von Fehlern \\
        H10          & Hilfe und Dokumentation                                       \\
        \uzlhline
    \end{tabular}
\end{table}

Aufgrund des zugrunde gelegten Entwurfs der EMI-Award-App wurde direkt mit einem
High-Fidelity Prototypen begonnen. Dieser wurde mit dem Interface-Design-Tool
Figma erstellt. Im Folgenden wird der fertige Entwurf präsentiert.

% Design System
% - Schriftgrößen
% - Schriftgewichtungen
% - Farblos -> Anpassbar
%
% Schriftart: Inter
% - Hohes Mittellängen-Schriftgrößen Verhältnis -> Sehr gut lesbar (DVSG)
%
% Schriftgröße: Laut DVSG Rechner mindestens 13px bei 30cm, 0.5 Visus, 0.7 MSV
% und 150 ppi
%
% Iconsatz: Heroicons



Aufgrund des variablen räumlichen Kontextes ist eine Verteilung der Stationen
über Kilometer große Räume möglich. Hierdurch ist die Distanz zu den
verschiedenen Stationen eine durchaus wichtige Information für Teilnehmende
(H1). Als Konsequenz wird die Distanz in verschiedenen relevanten Stellen des
User-Interfaces angezeigt, welche in den folgenden Absätzen näher erläutert
werden.

% - Icons im Popup, da klein und durch Veranstalter bestimmbar statt generischer
%   Planet
% - Durch Distanz + Icon an Info dazu, sehr viele Informationen -> Ausblenden
%   von Sekundär Informationen (H8)

\section{Frameworks}

\section{Systemarchitektur}

% Speichern von Nutzerdaten ohne konkrete Anmeldung

% Usability von interaktiven Karten
% - Alter
% - Behinderung
% - Technikaffinität

% Usability QR-Code Reader



%!%%%%%%%%%%!%
%! Frontend !%
%!%%%%%%%%%%!%
% inclusive Design
%   - Mehrsprachig
%   - A11y (Aria, W3C Empfehlungen)

% Auf "Refactoring UI"-Standards geachtet

% Tailwindcss Design-System verwendet

%!%%%%%%%%%!%
%! Backend !%
%!%%%%%%%%%!%
% Komplexe Modellierung: Gruppen / Einzel

% Darstellung der Datenbank Relation

% Strukturierung der API
%   - Gruppen API
%   - Besuchs API
%   - Benachrichtigungs API
%   - Abzeichen API

\section{Implikation für Implementierung}

% Gedanken zur technischen Umsetzung der EMI-App
%   - Vor-/Nachteile Web/Native
%   - Library Entscheidungen