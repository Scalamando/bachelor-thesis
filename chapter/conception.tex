\chapter{Konzeption}

In diesem Kapitel werden die Funktionen des Frameworks erarbeitet und
Designanforderungen festgehalten. Hierbei wurden vorhandene Evaluationen der
EMI-Award-App berücksichtigt und nach dem menschenzentrierten Gestaltungsprozess
vorgegangen. Aus den Designanforderungen wurden Konzepte für die verschiedenen
Ansichten und ihre Komponenten erstellt. Zudem wurden Konzepte für die benötigte
Infrastruktur und ihren Aufbau angefertigt.


\section{Vorhandene Konzeption der EMI-Award App}

% Tech-Stack
%   - Hat sich größten teils bewährt
%   - Web-App mit PWA Funktionalität

% Mockups
%   - Bereits getestet
%   - Grobe Struktur beibehalten

\section{Systemarchitektur}

% Stabilität und Zuverlässigkeit *sehr* wichtig
%   - Ausfall kann gesamtes Event lahmlegen
%   - Bei hoher Teilnehmeranzahl hohe Last

\section{Frontend}

% Auf "Refactoring UI"-Standards geachtet

% Tailwindcss Design-System verwendet

\section{Backend}

% Komplexe Modellierung: Gruppen / Einzel

% Darstellung der Datenbank Relation

% Strukturierung der API
%   - Gruppen API
%   - Besuchs API
%   - Benachrichtigungs API
%   - Abzeichen API