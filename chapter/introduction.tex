\chapter{Einleitung}

% TODO: Quellen überprüfen!

In den letzten 20 Jahren hat sich am Kern der Veranstaltungsorganisation wenig
verändert, jedoch bietet moderne Technik Veranstaltungen eine größere Bühne als
je zuvor. Digitale Medien haben in den letzten 10 Jahren den Zugang zu
Veranstaltungen enorm erleichtert \cite{Bladen2012}. Webinars,
Online-Konferenzen, -Training und -Workshops bieten Menschen quer über die Welt
verteilt die Möglichkeit, an Veranstaltungen teilzunehmen und dabei zu wählen,
was sie wann und wie sehen möchten. Jedoch fehlt reinen Online Veranstaltungen
ein wichtiger Aspekt von örtlich stattfindenden Veranstaltungen: die
Face-to-face Interaktion \cite{Bladen2012}. Soziale Interaktion ist ein wichtiges
psychologisches Grundbedürfnis \cite{Maslow1943}.

Anfang 2021 waren soziale Interaktionen durch die pandemische Lage nur sehr
eingeschränkt möglich. Zum EMI-Award 2021 wurde deshalb am Institut für
Multimediale und Interaktive Systeme (IMIS) der Universität zu Lübeck im Rahmen
eines Bachelor-Projekts die EMI-Award-App umgesetzt \cite{Canzler2021}. Beim
EMI-Award handelt es sich um eine jährliche Veranstaltung des Studiengangs
Medieninformatik, bei welcher „die kreativsten und gelungensten
Gruppenergebnisse aus der Erstsemester-Veranstaltung ‚Einführung in die
Medieninformatik‘ (EMI) ausgezeichnet [werden]“ \cite{UniversitatzuLubeck2021}.
Aufgrund der Kontaktbeschränkung konnten, im Gegensatz zu den vorherigen Jahren,
die Projekte der Studenten nicht im Foyer des Audimax an der Universität zu
Lübeck vorgestellt werden. Die App ermöglichte das Ansehen und Bewerten der im
gesamten Raum Lübeck verteilten Projekte. Innerhalb von 2 Wochen wurde die
Web-App von über 300 Personen genutzt. Teilnehmende erhielten die Möglichkeit,
die Projekte in ihrer eigenen Geschwindigkeit zu erkunden. Zudem ging aus
informellen Erfahrungsberichten hervor, dass Teilnehmende an Projektstandorten
öfter aufeinandertrafen.

% App, welche Veranstaltungen digital unterstüzt
%   - Nutzer einen roten Pfaden bieten
%       - Einführende Informationen zur Veranstaltung
%       - Informieren zu Standorten, Neuigkeiten
%       - Fortschrittsüberblick durch virtuelle Besuche / Einchecken
%       - Gruppen-Funktion
%       - soziale Interaktion durch Reaktionen
%       - Beteiligung durch Gamification
%       - Hilfe anbieten, häufige Probleme lösen
%   - Veranstalter die Informationsverteilung und -aufnahme stark vereinfachen
%       - Überblicken der Veranstaltung
%       - Spontan Feedback sammeln
%       - Jederzeitiges Eingreifen durch Benachrichtigungen
%       - Häufige Probleme dank FAQ abfangen
%       - Datensammlung für Evaluationszwecke
%   - Einfache Nutzung dank Web-App
%   - Keine Registration

\section{Ziele dieser Arbeit} \label{sec:goals}

Im Rahmen dieser Bachelorarbeit wurde eine Erweiterung und Verallgemeinerung der
EMI-Award-App in Form eines flexiblen AR-Event-App-Frameworks für ortsbasierte
Veranstaltungen konzipiert und entwickelt. Dabei wurde die Organisation von
Veranstaltungen und Teilnahme näher analysiert. Zudem wurden die Vor- und
Nachteile der bestehenden Anwendung einbezogen. Der Framework unterstützt
Veranstaltende ab der „Umsetzung“ und Teilnehmende ab der „Durchführung“ einer Veranstaltung.

Veranstaltende können über ein Online-Dashboard ihre Veranstaltung verwalten.
Dies umfasst das Anlegen von verschiedenen Standorten, Abzeichen und
Hilfseinträge. Des Weiteren sollen Push-Nachrichten sowie Feedbackanfragen an
alle Teilnehmenden abgeschickt werden können. Zusätzlich ermöglicht die Web-App
Daten zu besuchten Standorten, erhaltenen Abzeichen und Feedback gesammelt
darzustellen. Diese Daten können anschließend zur Auswertung der Veranstaltung
genutzt werden.

Teilnehmenden wird durch die übersichtliche Darstellung wichtiger Informationen
zur Veranstaltung ein roter Faden geboten. Dies geschieht mithilfe einer
Anzeige der Standorte, sowohl auf einer interaktiven Karte, als auch in
Listenform. Zudem können weitere Informationen und multimediale Inhalte zu den
einzelnen Standorten eingesehen werden. Das virtuelle Besuchen von Standorten
per QR-Code und Abzeichen sollen Teilnehmende zur Ausführung von bestimmten
gewünschten Aktionen lenken.

\section{Aufbau der Arbeit}

% Herczeg

Das Vorgehen dieser Arbeit basiert auf dem menschenzentrierten Gestaltungsprozess nach DIN
EN ISO 9241-210 (s. \autoref{fig:din-hcd}).

Im zweiten <blablabla...>

\begin{figure}[htpb]
    \renewcommand\baselinestretch{1}
    \centering
    \tikzset{
        textbox/.style={
                text width=4cm,
                text centered,
                align=center
            },
        arrow/.style={
                ->
            }
    }
    \tikz [thesis box shapes, baseline, anchor=base]{
        \node [block] (1) [textbox] at (-3, 8) {Den menschenzentrierten
            Gestaltungsprozess planen};
        \node [block] (2) [textbox] at (0, 3.75) {Den Nutzungskontext verstehen und
            beschreiben};
        \node [block] (3) [textbox] at (4.5, 0) {Die Nutzungsanforderungen
            spezifizieren};
        \node [block] (4) [textbox] at (0, -3) {Gestaltungslösungen entwickeln,
            die die Nutzungsanforderungen erfüllen};
        \node [block] (5) [textbox] at (-4.5, 0.25) {Gestaltungslösungen aus der
            Benutzerperspektive evaluieren};
        \node [block] (6) [textbox] at (-6, 4.5) {Gestaltungslösung
            erfüllt die Nutzungsanforderungen};
        \draw[arrow] (1.south) to [out=270, in=90] (2.north);
        \draw[arrow, bend left=45] (2.east) to (3.north);
        \draw[arrow, bend left=45] (3.south) to (4.east);
        \draw[arrow, bend left=45] (4.west) to (5.south);
        \draw[arrow] (5.west) to [out=180, in=270] (6.south);
        \draw[arrow, dashed, bend left=45] (5.north) to (2.west);
        \draw[arrow, dashed, bend left=35] (5.north) to (3.north);
        \draw[arrow, dashed, bend left=75] (5.north) to (4.north);
        \node [fill=white, fill opacity=0.7, text width=4cm, text opacity=1] (7) at (-3.5, 2) {Iteration, soweit
            Evaluationsergebnisse Bedarf hierfür aufzeigen};
    }
    \caption{Der menschenzentrierte Gestaltungsprozess \cite{iso-9241-210}}
    \label{fig:din-hcd}
\end{figure}









