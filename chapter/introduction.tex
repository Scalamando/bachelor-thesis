\chapter{Einleitung}

% Digitale Medien erleichtern den Zugang zu Veranstaltungen enorm
%   - Webinars, Online Konferenzen & Training & Workshops
%
% Jedoch: face-to-face Interaktion kann nicht ersetzt werden.
%   - Grundbedürfnis des Menschen (Maslow Pyramide)
%       - Soziale Interaktion
%       - Ritual
%
% Erfolg der EMI-Award-App


% App, welche Veranstaltungen digital unterstüzt
%   - Nutzer einen roten Pfaden bieten
%       - Einführende Informationen zur Veranstaltung
%       - Informieren zu Standorten, Neuigkeiten
%       - Fortschrittsüberblick durch virtuelle Besuche / Einchecken
%       - Gruppen-Funktion
%       - soziale Interaktion durch Reaktionen
%       - Beteiligung durch Gamification
%       - Hilfe anbieten, häufige Probleme lösen
%   - Veranstalter die Informationsverteilung und -aufnahme stark vereinfachen
%       - Überblicken der Veranstaltung
%       - Spontan Feedback sammeln
%       - Jederzeitiges Eingreifen durch Benachrichtigungen
%       - Häufige Probleme dank FAQ abfangen
%       - Datensammlung für Evaluationszwecke
%   - Einfache Nutzung dank Web-App
%   - Keine Registration

\section{Beiträge dieser Arbeit}

Im Rahmen dieser Bachelorarbeit wurde ein Framework konzipiert und
implementiert, welcher es ermöglicht, Veranstaltungen, welche sich über größere
Räume verteilen, mit einer Web-App zu unterstützen. Die Unterstützung gilt
Veranstaltenden sowie Teilnehmenden.

Teilnehmenden wird durch die übersichtliche Darstellung wichtiger Informationen
zur Veranstaltung ein roter Pfaden geboten. Dies geschieht mit Hilfe einer
übersichtlichen Anzeige der PoIs (Point of Interest), sowohl auf einer Karte,
als auch in Listenform. Zudem können weitere Informationen, inklusive
multimedialer Inhalte, zu den einzelnen PoIs eingesehen werden. Des Weiteren
ermöglichen das virtuelle Besuchen von PoIs per QR-Code und Abzeichen,
Teilnehmende zur Ausführung von bestimmten gewünschten Aktionen zu lenken.

Veranstaltende nutzen eine Web-Oberfläche, um die Veranstaltung zu erstellen und
zu verwalten. Hierzu zählt das Eintragen der PoIs, Abzeichen und Hilfseinträge,
sowie das Abschicken von Push-Benachrichtigung an alle Teilnehmenden. Zusätzlich
werden über die Web-App Daten zu besuchten PoIs, erhaltenen Abzeichen und
Feedback gesammelt. Diese Daten können anschließend zur Auswertung der
Veranstaltung genutzt werden.

\section{Verwandte Arbeiten}

Die Idee dieser Arbeit entstammt der im Rahmen eines Bachelorprojekts
entstandenen EMI-Award-App~\cite{Canzler2021}, welche bereits März 2021
eingesetzt wurde, um die verschiedenen studentischen Projekte des EMI-Awards in
der Stadt Lübeck auszustellen. Beim EMI-Award handelt es sich um eine jährliche
Veranstaltung des Studiengangs Medieninformatik, bei welcher die „die
kreativsten und gelungensten Gruppenergebnisse aus der
Erstsemester-Veranstaltung ‚Einführung in die Medieninformatik‘ (EMI)
ausgezeichnet [werden]“ \cite{UniversitatzuLubeck2021}. Die App ermöglichte
unter anderem das Einsehen, virtuelle Besuchen und Bewerten von verschiedenen
Stationen. Eine Besonderheit stellt die AR-Komponente der App dar. Nach dem
erfolgreichen Scannen des QR-Codes einer Station wurde zunächst, abhängig vom
Gerät des Nutzers, eine AR- beziehungsweise VR-Szene angezeigt, in welcher ein
Planet, symbolisch für die Station stehend, schwebte und angetippt werden
musste. Zudem konnten verschiedene Errungenschaften durch vorgegebene Aktionen
erhalten werden.
% Insbesondere in der Konzeptionsphase dieser Arbeit wurden die
% Evaluationsergebnisse der EMI-Award-App stark einbezogen und als Grundlage für
% den Prototyp genutzt.

\section{Aufbau der Arbeit}

Die Arbeit gliedert sich in einen theoretischen Teil, welcher die Grundlagen
erforscht und Anforderungen festlegt, und einen praktischen Teil, welcher die
Implementierung des Frameworks umfasst.









