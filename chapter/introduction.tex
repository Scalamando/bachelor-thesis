\chapter{Einleitung}

% Keine Veränderung in der Veranstaltungsorganisation in den letzten 20 Jahren
%
% Digitale Medien erleichtern den Zugang zu Veranstaltungen enorm
%   - Webinars, Online Konferenzen & Training & Workshops
%
% Jedoch: face-to-face Interaktion kann nicht ersetzt werden.
%   - Grundbedürfnis des Menschen (Maslow Pyramide)
%       - Soziale Interaktion

In den letzten 20 Jahren hat sich am Kern der Veranstaltungsorganisation wenig
verändert, jedoch bietet moderne Technik Veranstaltungen eine größere Bühne als
je zuvor. Digitale Medien haben in den letzten 10 Jahren den Zugang zu
Veranstaltungen enorm erleichtert \cite{Bladen2017}. Webinars, Online-
Konferenzen, -Training und -Workshops bieten Menschen quer über die Welt verteilt
die Möglichkeit, an Veranstaltungen teilzunehmen und dabei zu wählen, was sie
wann und wie sehen möchten. Oft werden hier große Plattformen genutzt, um die
Infrastruktur aufzubauen. Jedoch fehlt reinen Online Veranstaltungen ein
wichtiger Aspekt von physisch stattfindenden Veranstaltungen: die Face-to-face
Interaktion. Diese Interaktion bedient viele der psychologischen menschlichen
Grundbedürfnisse \cite{Maslow1943} und stellt somit einen bedeutenden Aspekt
von Veranstaltungen dar.

% Erfolg der EMI-Award-App
%
% Coroni


% App, welche Veranstaltungen digital unterstüzt
%   - Nutzer einen roten Pfaden bieten
%       - Einführende Informationen zur Veranstaltung
%       - Informieren zu Standorten, Neuigkeiten
%       - Fortschrittsüberblick durch virtuelle Besuche / Einchecken
%       - Gruppen-Funktion
%       - soziale Interaktion durch Reaktionen
%       - Beteiligung durch Gamification
%       - Hilfe anbieten, häufige Probleme lösen
%   - Veranstalter die Informationsverteilung und -aufnahme stark vereinfachen
%       - Überblicken der Veranstaltung
%       - Spontan Feedback sammeln
%       - Jederzeitiges Eingreifen durch Benachrichtigungen
%       - Häufige Probleme dank FAQ abfangen
%       - Datensammlung für Evaluationszwecke
%   - Einfache Nutzung dank Web-App
%   - Keine Registration

\section{Ziele dieser Arbeit} \label{sec:goals}

Im Rahmen dieser Bachelorarbeit wurde ein Framework konzipiert und
implementiert, welcher es ermöglicht, Veranstaltungen, welche sich über größere
Räume verteilen, mit einer Web-App zu unterstützen. Die Unterstützung gilt
Veranstaltenden sowie Teilnehmenden.

Teilnehmenden wird durch die übersichtliche Darstellung wichtiger Informationen
zur Veranstaltung ein roter Pfaden geboten. Dies geschieht mit Hilfe einer
übersichtlichen Anzeige der PoIs (Point of Interest), sowohl auf einer Karte,
als auch in Listenform. Zudem können weitere Informationen, inklusive
multimedialer Inhalte, zu den einzelnen PoIs eingesehen werden. Des Weiteren
ermöglichen das virtuelle Besuchen von PoIs per QR-Code und Abzeichen,
Teilnehmende zur Ausführung von bestimmten gewünschten Aktionen zu lenken.

Veranstaltende nutzen eine Web-Oberfläche, um die Veranstaltung zu erstellen und
zu verwalten. Hierzu zählt das Eintragen der PoIs, Abzeichen und Hilfseinträge,
sowie das Abschicken von Push-Benachrichtigung an alle Teilnehmenden. Zusätzlich
werden über die Web-App Daten zu besuchten PoIs, erhaltenen Abzeichen und
Feedback gesammelt. Diese Daten können anschließend zur Auswertung der
Veranstaltung genutzt werden.

\section{Aufbau der Arbeit}

Die Arbeit gliedert sich in einen theoretischen Teil, welcher die Grundlagen
erforscht und Anforderungen festlegt, und einen praktischen Teil, welcher die
Implementierung des Frameworks umfasst.









