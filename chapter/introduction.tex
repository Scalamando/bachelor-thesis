\chapter{Einleitung}



\section{Beiträge dieser Arbeit}

Im Rahmen dieser Bachelorarbeit wurde ein Framework konzipiert und
implementiert, welcher es ermöglicht, Veranstaltungen, welche sich über größere
Räume verteilen, mit einer Web-App zu unterstützen. Die Unterstützung gilt
Veranstaltenden sowie Teilnehmenden.

Teilnehmenden wird durch die übersichtliche Darstellung wichtiger Informationen
zur Veranstaltung ein roter Pfaden geboten. Dies geschieht mit Hilfe einer
übersichtlichen Anzeige der PoIs (Point of Interest), sowohl auf einer Karte,
als auch in Listenform. Zudem können weitere Informationen, inklusive
multimedialer Inhalte, zu den einzelnen PoIs eingesehen werden. Des Weiteren
ermöglichen das virtuelle Besuchen von PoIs per QR-Code und Abzeichen,
Teilnehmende zur Ausführung von bestimmten gewünschten Aktionen zu lenken.

Veranstaltende nutzen eine Web-Oberfläche, um die Veranstaltung zu erstellen und
zu verwalten. Hierzu zählt das Eintragen der PoIs, Abzeichen und Hilfseinträge,
sowie das Abschicken von Push-Benachrichtigung an alle Teilnehmenden. Zusätzlich
werden über die Web-App Daten zu besuchten PoIs, erhaltenen Abzeichen und
Feedback gesammelt. Diese Daten können anschließend zur Auswertung der
Veranstaltung genutzt werden.

\section{Verwandte Arbeiten}

Die Idee dieser Arbeit entstammt der im Rahmen eines Bachelorprojekts
entstandenen EMI-Award-App~\cite{Canzler2021}, welche bereits März 2021
eingesetzt wurde, um die verschiedenen studentischen Projekte des EMI-Awards in
der Stadt Lübeck auszustellen. Beim EMI-Award handelt es sich um eine jährliche
Veranstaltung des Studiengangs Medieninformatik, bei welcher die „die
kreativsten und gelungensten Gruppenergebnisse aus der
Erstsemester-Veranstaltung ‚Einführung in die Medieninformatik‘ (EMI)
ausgezeichnet [werden]“ \cite{UniversitatzuLubeck2021}. Die App ermöglichte das
unter anderem das Einsehen, virtuelle Besuchen und Bewerten von verschiedenen
Stationen. Eine Besonderheit stellt die AR-Komponente der App dar. Nach dem
erfolgreichen Scannen des QR-Codes einer Station wurde zunächst, abhängig vom
Gerät des Nutzers, eine AR- beziehungsweise VR-Szene angezeigt, in welcher ein
Planet, symbolisch für die Station stehend, schwebte und angetippt werden
musste. Zudem konnten verschiedene Errungenschaften durch bestimmte Aktionen erhalten werden. Insbesondere in der Konzeptionsphase dieser Arbeit wurden die
Evaluationsergebnisse der EMI-Award-App stark einbezogen und als Grundlage für
den Prototyp genutzt.

\section{Aufbau der Arbeit}

Zu Beginn der Arbeit wurde eine Literaturrecherche durchgeführt. In dieser
wurden vergleichbare Projekte, sowie grundlegende Konzepte für Veranstaltungen
gesucht. Gleichzeitig wurden die Ergebnisse der Evaluation der EMI-Award-App
gesichtet, um die dort gewonnen Erkenntnisse zur Implementierung von Web-Apps
sowie deren Einsatz zu erhalten. Daraufhin wurden Interviews mit teilnehmenden
und veranstaltenden Personen durchgeführt. Die Durchführung mit veranstaltenden
Personen erfasste relevante Veranstaltungstypen, sowie wertvolle Funktionen aus
organisatorischer Sicht. Hingegen erfassten die Interviews mit teilnehmenden
Personen Erfahrungen mit digital unterstützen Veranstaltungen und hilfreiche
vorhandene und gewünschte Funktionen.

Anschließend wurden die Funktionen des Frameworks erarbeitet und
Designanforderungen festgehalten. Hierbei wurden vorhandene Evaluationen der
EMI-Award-App berücksichtigt und nach dem menschenzentrierten Gestaltungsprozess
vorgegangen. Aus den Designanforderungen wurden Konzepte für die verschiedenen
Ansichten und ihre Komponenten erstellt. Zudem wurden Konzepte für die benötigte
Infrastruktur und ihren Aufbau angefertigt.

Unterstützt durch die erstellen Konzepte wurden Frontend und Backend der
Anwendung implementiert. Um die Softwarequalität bei der gegebenen Komplexität
des Frameworks garantieren zu können, wurden Softwaretests eingesetzt, welche
die grundlegende Logik der verschiedenen Funktionen des Backends überprüfen
sollten. Des Weiteren wurde ein hohes Maß an Automation angestrebt, um die
Softwarequalität kontinuierlich zu überprüfen und die verwendete Zeit für
repetitive Aufgaben zu minimieren.

Darauf folgend wurde der Framework im Rahmen der Stadt- und Campusrallye der
Vorwoche an der Universität zu Lübeck eingesetzt und anschließend evaluiert.
Hierzu wurde für Teilnehmende eine quantitative Online-Befragung durchgeführt,
während die Veranstaltenden im Anschluss zum Einsatz interviewt wurden.
