\chapter{Analyse}

In diesem Kapitel wurden die betrachteten Veranstaltungen zunächst eingegrenzt.
Anschließend wurde eine Organisationsanalyse durchgeführt, um die verschiedenen
Phasen einer Veranstaltung einzubeziehen. Die Phasen „Umsetzung“, „Durchführung“
und „Schließung“ wurden hierbei näher analysiert. Basierend auf den Ergebnissen
der Organisationsanalyse wurden eine Benutzer- und Kontextanalyse durchgeführt.
Die Zielgruppen des Frameworks wurden festgelegt und für die Konzeption
relevante Besonderheiten herausgearbeitet. In einer abschließenden
Kontextanalyse wurden die verschiedenen Kontexte der Nutzung des Frameworks
aufgezeigt und auf die mit ihnen einhergehenden Herausforderungen eingegangen.


\section{Datenquellen}

Im Rahmen der Analyse wurden verschiedene Datenquellen genutzt. Eine
wissenschaftliche Literaturrecherche wurde zwischen dem 26.07.2021 und
11.08.2021 durchgeführt. Verschiedene vergleichbare Projekte wurden im Rahmen
einer Internetrecherche ermittelt. Die Ausarbeitung und Evaluation der
EMI-Award-App wurde als Quelle genutzt. Zudem wurden Interviews mit
Veranstaltenden und Teilnehmenden durchgeführt. Die Interviews fanden zwischen
dem 5.09.2021 und 17.09.2021 statt. \\
Zur Literaturrecherche wurden die digitalen Bibliotheken ACM Digital Library
(\url{https://dl.acm.org/}) und Google Scholar
(\url{https://scholar.google.de/}) genutzt. Es wurden Begriffe aus den Bereichen
Event Management (\emph{Event Management, Event Evaluation, Event Experience,
    Event Tracking}) und UX (\emph{Attention Economy, Attention Management,
    Information Overload, Digital Burnout, Overchoice, Web Animation, Animation UX,
    UX Micro Interaction, Animation Usability}) zur Suche verwendet. \\
Die Interviews mit Veranstaltenden und Teilnehmenden wurden qualitativ und
semistrukturiert, mit Hilfe eines dafür entworfenen Interviewleitfadens (siehe
Anhang A), durchgeführt. Bei den Veranstaltenden handelt es sich um Personen mit
praktischer Erfahrung in der Organisation von Veranstaltungen. In Tabelle
\ref{table:ver-soz} und Tabelle \ref{table:teil-soz} sind Details zu den
befragten Personen aufgelistet. Die IDs der interviewten Personen werden im
Verlauf der Analyse referenziert.

\begin{table}[htpb]
    \def\arraystretch{1.25}
    \centering
    \caption{Soziodemografische Daten der Veranstaltenden}
    \label{table:ver-soz}
    \begin{tabular}{lcll}
        \uzlhline
        \uzlemph{ID} & \uzlemph{Geschlecht} & \uzlemph{Alter} &
        \uzlemph{Vorerfahrung}                                                \\
        \uzlhline V1 & w                    & 40 - 59 J.      & Verschiedene
        Veranstaltungen, verteilt über 11 Jahre                               \\
        V2           & m                    & 18 - 25 J.      & Mehrere große
        Veranstaltung (>2000 Teilnehmende)                                    \\
        \uzlhline
    \end{tabular}
\end{table}

\begin{table}[htpb]
    \def\arraystretch{1.25}
    \centering
    \caption{Soziodemografische Daten der Teilnehmenden}
    \label{table:teil-soz}
    \begin{tabular}{lclcl}
        \uzlhline
        \uzlemph{ID}                     & \uzlemph{Geschlecht} & \uzlemph{Alter} &
        \uzlemph{EMI-Award-App genutzt?} & \uzlemph{Tätigkeit}                        \\
        \uzlhline T1                     & w                    & 18 - 25 J.      & j
                                         & Student:in                                 \\
        T2                               & m                    & 18 - 25 J.      & j
                                         & Student:in                                 \\
        T3                               & w                    & 18 - 25 J.      & j
                                         & Student:in                                 \\
        T4                               & m                    & 18 - 25 J.      & n
                                         & Student:in                                 \\
        T5                               & w                    & 40 - 59 J.      & n
                                         & Berufstätig                                \\
        \uzlhline
    \end{tabular}
\end{table}


\section{Definition: Veranstaltung} \label{sec:analysis-def}

Bevor die verschiedenen Aufgaben, Probleme und Benutzergruppen analysiert werden
können, müssen Umfang und Typ der Veranstaltung festgelegt werden. In der
Literatur sind verschiedene Definitionen zum Begriff „Veranstaltungen“ zu
finden, welche sich in Umfang und Inhalt unterscheiden. Die räumliche und
zeitliche Dimension von Veranstaltungen ist sehr flexibel. \textcite{Getz2007}
definiert Veranstaltungen als zeitliche Phänomene. Bei geplanten Veranstaltungen
wird das Veranstaltungsprogramm oder der Zeitplan im Allgemeinen detailliert
geplant und im Voraus gut bekannt gemacht. Geplante Veranstaltungen sind in der
Regel auch auf bestimmte Orte beschränkt, wobei es sich um eine bestimmte
Einrichtung, eine sehr große Freifläche oder viele Orte handeln kann.
\textcite{Bladen2017} definieren Veranstaltungen als zeitlich begrenzte und
zweckgebundene Zusammenkünfte von Menschen definiert.

Im Rahmen dieser Arbeit werden Veranstaltungen betrachtet, welche in ihrer
zeitlichen Dimension uneingeschränkt sind, jedoch räumlich verteilt stattfinden.
Konkret können die betrachteten Veranstaltungen wenige Stunden, Wochen oder ohne
festgelegtes Ende stattfinden. Zudem verteilen sich die Aktivitäten der
Veranstaltung auf einen größeren Raum. Dies kann mindestens die Aufteilung auf
verschiedene Räumlichkeiten sein, bis hin zur Verteilung über verschiedene
Stadtteile.


\section{Organisationsanalyse} \label{sec:analysis-org}

% Masterarbeit_Holtz  Aufgabenanalyse!!!
Die hier analysierte Vorgehensweise in der Organisation von Veranstaltungen
basiert auf dem EMBOK-Model \cite{Silvers2013} und den Aussagen der interviewten
Veranstaltenden. Typischerweise lässt sich die Organisation von Veranstaltungen
in 5 Phasen (s. \autoref{fig:embok-phases}) gliedern: Initiation, Planung,
Umsetzung, Durchführung und Schließung. Die Vorgehensweise in den Phasen
gestaltet sich wie folgt:

\begin{enumerate}
    \setlength{\itemsep}{1em}
    \item \textbf{Initiation} \\
          Es wird geforscht, um ein Konzept zu erstellen und zu validieren.
          Umfang und Kontext sowie Ziele und Aufgaben werden festgesetzt.
    \item \textbf{Planung} \\
          Anforderungen und Spezifikationen werden festgehalten. Hierzu zählen
          die stattfindenden Aktivitäten, sowie die Art der Organisation und
          erforderliche Ressourcen.
    \item \textbf{Umsetzung} \\
          Alle für die Veranstaltung benötigten Waren und Dienstleistungen
          werden in Auftrag gegeben und koordiniert. Der Fokus liegt unter
          anderem auf der Überwachung und Überprüfung des Umfangs und Zeitplans,
          sowie der Kosten und Qualität.
    \item \textbf{Durchführung} \\
          Die veranstaltungsbezogenen Aktivitäten werden aufgenommen. Ab Beginn
          dieser Phase ist der Handlungsrahmen stark eingeschränkt, wodurch der
          Fokus auf der Überwachung der Veranstaltung liegt.
    \item \textbf{Schließung} \\
          Nach Abschluss der Veranstaltung werden Daten gesammelt, ausgewertet
          und weitergegeben. Das Ziel ist die Dokumentation der Erkenntnisse für
          weitere Veranstaltungen.
\end{enumerate}

\begin{figure}[htpb]
    \centering
    \includegraphics[width=\textwidth]{embok_phases.jpg}
    \caption{Die Phasen der Veranstaltungsorganisation \cite{Silvers2013b}. Die
        verschiedenen Phasen (vertikal) bilden die Grundlage einer
        Veranstaltung. Verwoben sind diese mit Wissensdomänen (horizontal).
        Sollte einer der Fäden verschwinden, wird das gesamte Geflecht
        geschwächt.}
    \label{fig:embok-phases}
\end{figure}

% TODO: !Wissensdomänen erklären!

Da das Ziel dieser Arbeit die Unterstützung ab der Umsetzung ist (s.
\autoref{sec:goals}), werden die Phasen „Umsetzung“, „Durchführung“ und
„Schließung“ näher betrachtet.

\subsection{Umsetzung} \label{sec:analysis-org-umsetzung}

In dieser Phase werden die herausgearbeiteten Pläne umgesetzt. Hierbei werden
Umfang, Zeitpläne, Kosten, Kommunikation und Risiken überwacht und kontrolliert,
um die plangemäße Ausführung sicherzustellen. Weitere Arbeits-/Hilfskräfte
werden engagiert, um bei der Umsetzung mitzuwirken. Für den Erfolg der
Veranstaltung ist die effektive Kommunikation wichtig. Um diese zu
gewährleisten, muss eine Kommunikationsinfrastruktur für Organisierende,
Mitwirkende und externe Dienstleister eingeführt werden. Hilfskräfte müssen in
die Kommunikationsinfrastruktur eingewiesen werden. Aufgrund von mangelnden
technischen Fähigkeiten können bei komplexeren Infrastrukturen Probleme
auftreten (V2).

\subsection{Durchführung} \label{sec:analysis-org-durchfuehrung}

Mit Beginn der Durchführung ändert sich die Dynamik der Veranstaltung bedeutend.
Aufgrund der Anwesenheit von Teilnehmenden sind tiefgreifende Änderungen nun
nicht mehr möglich und beschränken sich auf die Behebung von kleinen Problemen
(V2). Hingegen wird die wichtigste Aufgabe die Überwachung der Veranstaltung.
Unter besonderer Beobachtung stehen logistische Tätigkeiten, sowie unerwartet
auftretende Probleme (V1, V2). Diese können u. a. durch Teilnehmende,
stattfindende Aktivitäten oder die Umgebung der Veranstaltung ausgelöst werden.
\\
Während Probleme, welche die Durchführung der Veranstaltung direkt
beeinträchtigen könnten, streng beobachtet und behoben werden, stehen die
Interaktionen und Erfahrungen der Teilnehmenden zunächst im Hintergrund.
Abgesehen von kritischem Feedback, wird die Erfahrung erst während der
Schließung erfasst und ausgewertet (V2).

\subsection{Schließung} \label{sec:analysis-org-schliessung}

% Bedeutung Evaluation aus Finischen Case S.57-59 Mischung Quantitativ &
% Qualitativ Evaluation S.61

Zur Schließung der Veranstaltung wird die Evaluation zur bedeutendsten Aufgabe.
Feedback und Daten werden von organisierender Seite gesammelt. Hierbei wird
zwischen \textit{weichen} und \textit{harten Daten} unterschieden. Harte Daten
bestehen aus Teilnehmerzahlen, Andrang an verschiedenen Aktivitäten, Dauer,
Einkommen und weiteren zählbaren Merkmalen. Im Vergleich dazu bestehen weiche
Daten u. a. aus Beschwerden, Konflikten, Problemen, Komplimenten, Reaktionen und
Empfehlungen. Mitwirkende, Hilfskräfte und die Organisierenden werden meist dazu
in einer Nachbesprechung befragt (V1,V2). Das Feedback von Teilnehmenden wird
oft nur passiv vor Ort vom Personal erfasst und weitergegeben oder beschränkt
sich auf Familie und Freunde (V2). Die erfassten Erkenntnisse werden
dokumentiert, um auf das Wissen in der nächsten Veranstaltung zurückgreifen zu
können.


\section{Benutzeranalyse} \label{sec:analysis-user}

Um die Funktionen des Frameworks zielgruppengerecht gestalten zu können, werden
in diesem Abschnitt die Benutzergruppen des Frameworks festgelegt und näher
analysiert. Zu den Benutzergruppen gehören Veranstaltende, sowie Teilnehmende
einer Veranstaltung.

\subsection{Veranstaltende}

Inzwischen existieren weltweit tausende Einrichtungen, welche formale
Qualifikationen und Ausbildungen im Veranstaltungsmanagement anbieten. Jedoch
sind diese Qualifikationen nicht einheitlich festgelegt, was zu
unterschiedlichen Schwerpunkten, Umfang, Vermittlungsart und letztendlich
erworbenen Qualifikationen führt \cite{Bladen2017}. In Deutschland gibt es den
anerkannten Ausbildungsberuf des/der Veranstaltungs\-kaufmann/-frau. Die
Aufgaben sind hierbei die Konzipierung und Organisation des kaufmännischen
Aspektes von Veranstaltungen \cite{Kultusministerkonferenz2001}. Die Aufgaben
eines/r Event Manager/in umfassen zusätzlich die allgemeine Organisation und
Aufgaben im Marketing \cite{BundesagenturfurArbeit2021}. Jedoch ist die
Berufsbezeichnung „Event Manager/in“ rechtlich nicht geschützt. \\
Zudem bedarf es in Deutschland keiner formalen Qualifikation, um eine
Veranstaltung beliebiger Größe zu organisieren. Dementsprechend können keine
bestimmten Qualifikationen für Veranstaltende angenommen werden.

\subsection{Teilnehmende}

Die Teilnehmenden einer Veranstaltung unterscheiden sich stark in ihren
soziodemografischen Daten je nach Typ der Veranstaltung. Im Kontext der Arbeit
sind besonders Alter, Technikaffinität und körperliche, seelische, geistige oder
Sinnesbeeinträchtigungen. Für die Betrachtung in dieser Arbeit wird das Alter
der Teilnehmenden auf 16 - 60 J. begrenzt. Zudem wird die sichere grundlegende
Umgangsweise mit einem Smartphone vorausgesetzt.

Da die Literatur zu Event Management ihren Fokus auf die Organisation legt, sind
nur wenig Informationen zur Sicht der Teilnehmenden und ihren Hürden vorhanden.
Als Messwerte für eine erfolgreiche Veranstaltung werden meist
geschäftsrelevante Werte wie z. B. verkaufte Tickets, Teilnehmerzahlen, Umsatz
oder öffentliche Aufmerksamkeit verwendet, welche wenig über die Erfahrung der
Teilnehmenden aussagen. In Interviews wurden die Teilnehmenden gebeten zu
begründen, was eine gewählte Veranstaltung zu ihrem persönlichen Favoriten macht
(s. \autoref{table:teil-fav}).

\begin{table}[htpb]
    \def\arraystretch{1.25}
    \centering
    \caption{Merkmale von positiv empfundenen Veranstaltungen}
    \label{table:teil-fav}
    \begin{tabular}{ll}
        \uzlhline
        \uzlemph{Grund}               & \uzlemph{ID} \\
        \uzlhline  Sozialer Austausch & T1, T2       \\
        Innovative Technik            & T2           \\
        Persönliche Bindung           & T4           \\
        Gute Musik                    & T3           \\
        Einfache Wegfindung           & T5           \\
        Neues Wissen                  & T5           \\
        \uzlhline
    \end{tabular}
\end{table}

\section{Kontextanalyse} \label{sec:analysis-context}

In diesem Abschnitt werden der zeitliche und räumliche Kontext untersucht. Aus
der Beschreibung in \autoref{sec:goals} geht der Fokus auf die Unterstützung von
Veranstaltenden und Teilnehmenden hervor. Die Unterteilung der Benutzergruppen,
wie in \autoref{sec:analysis-user} beschrieben, ist zu beachten. Hieraus ergeben
sich unterschiedliche räumliche und zeitliche Kontexte für Veranstaltende und
Teilnehmende.

Die Unterstützung der Veranstaltenden findet in verschiedene Phasen der
Organisation statt. Der zeitliche Kontext kann mit Blick auf
\autoref{sec:analysis-org} in vor, während und nach einer Veranstaltung
unterteilt werden. Vor der Durchführung einer Veranstaltung ist insbesondere die
Vorbereitung von Aktivitäten und Absicherung von Dienstleistungen oder Waren
wichtig. Während der Veranstaltung hingegen findet ein starker Fokuswechsel auf
die Beobachtung und schnelle Behebung von Problemen statt. Nach einer
Veranstaltung sind die Datensammlung und Auswertung, sowie der Wissenstransfer
von Bedeutung. \\
Auch beschränken sich die betrachteten Veranstaltungen während der Durchführung
auf den in \autoref{sec:analysis-def} beschriebenen örtlichen Bereich, woraus
sich ein weites gehend unbeschränkter räumlicher Kontext ergibt. Vor und nach
der Veranstaltung werden komplexe Informationen eingetragen oder ausgewertet. Da
die Bildschirmgröße einen großen Einfluss auf die effiziente Verarbeitung von
Informationen hat, wird hierfür von einem Arbeitsplatz mit großem Bildschirm
ausgegangen. Somit ergibt sich ein räumlich eingeschränkter Kontext vor und nach
der Veranstaltung. Folglich wird während der Veranstaltung ein mobiles System
zur Überwachung benötigt, wobei die Bedingungen vor und nach der Veranstaltung
ein stationäres System befürworten.

Für Teilnehmende ist der Kontext ebenfalls in vor, während und nach der
Veranstaltung aufgeteilt. Der zeitliche Kontext während der Veranstaltung ist
hierbei am bedeutendsten. Auch für Teilnehmende gilt der in
\autoref{sec:analysis-def} beschriebene örtliche Bereich, woraus sich ebenfalls
ein unbeschränkter räumlicher Kontext ergibt. Folglich muss das System für die
Teilnehmenden mobil einsetzbar sein.


\section{Analyse des bestehenden Systems}

An dieser Stelle wird die Implementation und Evaluation der
\textit{EMI-Award-App} vorgestellt, welche die Grundlage des neuen Systems bildet.

% Tech-Stack
%   - Hat sich größten teils bewährt
%   - Web-App mit PWA Funktionalität

% Mockups
%   - Bereits getestet
%   - Grobe Struktur beibehalten

Die \textit{EMI-Award-App} ermöglichte das augmentierte Besuchen, Einsehen und
Bewerten der EMI-Award-Projekte und bildet die Grundlage dieser Arbeit (s.
\autoref{sec:goals}). Speziell wurden dafür an verschiedenen Standorten Schilder
mit QR-Codes angebracht, welche mit einem QR-Code-Reader oder der App gescannt
werden konnten (s. \autoref{fig:emi-qr-code}).

\begin{figure}[htpb]
    \centering
    \includegraphics{emi_schild.png}
    \caption{Schild eines Projektstandortes mit QR-Code}
    \label{fig:emi-qr-code}
\end{figure}

Beim ersten Aufruf der App wird dem Nutzenden eine kleine Einführung in die App
und ihren Hintergrund gegeben. Dies geschieht über eine interaktive Slideshow.
Nach dem Bestätigen der letzten Slide werden Nutzende zur interaktiven Karte
geführt (s. \autoref{fig:emi-intro-map}). Auf dieser werden die Projektstandorte
durch verschiedenfarbige Icons von Planeten (\textgraphics{emi_planet.png})
dargestellt. Durch das Antippen eines Standorts wurde eine Kurzinformation, in
Form eines Pop-ups, zum jeweiligen Projekt angezeigt. Außerdem wurde der
Standort des Nutzenden angezeigt, insofern die Berechtigungen dazu gegeben
wurden.

\begin{figure}[htpb]
    \begin{minipage}{.5\textwidth}
        \centering
        \includegraphics[width=.6\linewidth]{emi_prolog.png}
    \end{minipage}%
    \begin{minipage}{.5\textwidth}
        \centering
        \includegraphics[width=.6\linewidth]{emi_map.png}
    \end{minipage}
    \caption{Einführung und interaktive Karte mit Projektstandorten}
    \label{fig:emi-intro-map}
\end{figure}

Um die vollständigen Informationen eines Projekts einzusehen, musste der
Standort erst virtuell besucht werden. Dies geschah über das Scannen des
QR-Codes auf dem Schild des Standorts (s. \autoref{fig:emi-qr-code}) oder der
manuellen Eingabe eines spezifischen Codes. Nach erfolgreichem Scannen oder
Eingeben des Codes wurde eine virtuelle Szene angezeigt. In dieser musste ein
Planet gesucht und angetippt werden, um den Vorgang abzuschließen (s.
\autoref{fig:emi-ar}). Je nach benutztem Gerät wurde die Szene per Augmented
Reality (AR) oder Virtual Reality angezeigt (VR). Zudem wurde die Anzahl der
besuchten Standorte in der Karten- und Projektansicht anhand eines
Fortschrittsbalkens angezeigt. Die vollständigen Informationen der Projekte
enthielten mediale Inhalte, eine detaillierte Beschreibung und die Autoren des
Projekts. Zusätzlich konnte das Projekt mit einer vorgegebenen Auswahl bewertet
werden (s. \autoref{fig:emi-project}).

\begin{figure}[htpb]
    \begin{minipage}{.5\textwidth}
        \centering
        \includegraphics[width=.6\linewidth]{emi_ar-1.png}
    \end{minipage}%
    \begin{minipage}{.5\textwidth}
        \centering
        \includegraphics[width=.6\linewidth]{emi_ar-2.png}
    \end{minipage}
    \caption{AR/VR Szene während des virtuellen Besuchs}
    \label{fig:emi-ar}
\end{figure}

\begin{figure}[htpb]
    \begin{minipage}{.5\textwidth}
        \centering
        \includegraphics[width=.6\linewidth]{emi_project-1.png}
    \end{minipage}%
    \begin{minipage}{.5\textwidth}
        \centering
        \includegraphics[width=.6\linewidth]{emi_project-2.png}
    \end{minipage}
    \caption{Einzelansicht eines Projekts}
    \label{fig:emi-project}
\end{figure}

Des Weiteren konnten verschiedene Abzeichen für bestimmte Aktionen erhalten
werden. Aktionen waren z. B. das Besuchen einer bestimmten Anzahl von Projekten
oder Benutzen von bestimmten Funktionen. Alle Abzeichen und ihr Fortschritt
konnten in einer Übersicht eingesehen werden (s. \autoref{fig:emi-achievments}).
Bestimmte Abzeichen wurden in mehreren Stufen freigeschaltet oder blieben bis
zum Erhalt verborgen. \\
Zudem wurden die gesammelten „Awardteile“ angezeigt. Awardteile sind sammelbare
Objekte, welche auf der Karte mit einem Meteorit-Symbol
(\textgraphics{emi_meteorit.png}) gekennzeichnet waren. Beim Annähern an die
gezeigten Standorte wurde eine Schaltfläche hervorgehoben, welche durch Antippen
das entsprechende Awardteil sammelt.

\begin{figure}[htpb]
    \centering
    \includegraphics[width=.3\textwidth]{emi_achievements.png}
    \caption{Abzeichen- und Awardteilansicht}
    \label{fig:emi-achievments}
\end{figure}

Außerdem werden die genannten Funktionen in der App durch eine Slideshow
erklärt, welche jederzeit über die Navigationsleiste erreichbar ist (s.
\autoref{fig:emi-help}).

\begin{figure}[htpb]
    \centering
    \includegraphics[width=.3\textwidth]{emi_help.png}
    \caption{Eine Seite der In-App-Anleitung}
    \label{fig:emi-help}
\end{figure}

Die EMI-Award-App wurde als progressive Web-App (PWA) realisiert und war unter
\url{https://app.emi-award.de} verfügbar.


\section{Problemanalyse} \label{sec:analysis-problems}

Während Planung, Durchführung und Nachbereitung von Veranstaltungen können
Probleme auftreten, welche die Organisation erschweren. Ähnlich treten bei der
Teilnahme Probleme auf, welche die Erfahrung beeinträchtigen können.

\subsection{Organisation} \label{sec:analysis-problems-orga}

In der Organisation von Veranstaltungen treten vor allem Probleme in der
Kommunikation auf. Je nach Größe der Veranstaltung können eine Vielzahl an
Menschen an der Organisation beteiligt sein. Meist sind verschiedene digitale
Tools oder Plattformen zur Kommunikation erforderlich, um sich mit allen
Veranstaltenden zu verständigen (V2). Hier kann es zu Herausforderungen kommen,
auch Menschen mit geringem Technikverständnis einzuweisen (V2). Aber auch der
Kontakt zu Teilnehmenden spielt eine große Rolle. Jedoch ist der Kontakt während
einer Veranstaltung nur gering (V1, V2). Nach einer Veranstaltung fällt der
Kontakt ebenfalls sehr schwer (V1) oder wird komplett ausgelassen (V2).

Weitere Probleme treten in der Beeinflussung von Veranstaltungen während ihrer
Durchführung auf. Es besteht der Bedarf kleine Änderungen noch einfacher
vornehmen zu können (V1). Zudem mangelt es bei räumlich verteilten
Veranstaltungen an Möglichkeiten ein Gruppengefühl zu erzeugen (V1).

Außerdem treten Probleme in der Übersicht über die Veranstaltung auf. Gerade bei
räumlich größeren Veranstaltungen gibt es keine direkte Übersicht über die
Teilnehmenden (V1). Oft werden für die Datensammlung nur Schätzungen verwendet
(V2). Jedoch ist die Reichweite ein wichtiger Indikator für Veranstaltende (V1).
Allgemein besteht die Datensammlung in der Nachbesprechung aus Aussagen von
Veranstaltenden oder deren Angehörigen und Freunden (V2).

\subsection{Teilnahme}
Durch die Interviews mit Teilnehmenden konnte festgestellt werden, dass
Schwierigkeiten in der Navigation von Veranstaltungen zu finden sind. Daher
besteht die Nachfrage nach einer Unterstützung in der Wegfindung und eine
Übersicht über die Veranstaltung (Person xy). Diese Unterstützung kann zum
Beispiel ein Event-Assistent umfassen (Person xy).
Des Weiteren konnte sowohl durch Teilnehmende, als auch durch Veranstaltende
erfasst werden, dass Personen mit körperlichen Einschränkungen durch technische
Unterstützung besser oder womöglich überhaupt an Events teilnehmen können
(bspw. Gebärdensprache in Form von Videos, Videos für Blinde, etc.).


\section{Formalisierte Anforderungen} \label{sec:analysis-implic}

In diesem Abschnitt werden die systematisch formalisierten Anforderungen
dargelegt, welche die Ergebnisse der gesamten Analyse nach
\textcite{Balzert2009} zusammenfassen. Zu Beginn werden Visionen und darauf
aufbauende Ziele definiert. Zudem werden Rahmenbedingungen und Kontext des
Systems aufgezeigt. Anschließend werden die daraus resultierenden funktionalen
und qualitativen Anforderungen festgelegt.

\subsection{Vision und Ziele}

Die Grundlage der Anforderungen bilden die Visionen und Ziele des Systems. Das
Festlegen dieser erlaubt eine Überprüfung der Zielgerichtetheit der
Anforderungen \cite{Balzert2009}. Diese leiten sich aus den Zielen der Arbeit
(\autoref{sec:goals}), sowie der vorangegangenen Analyse der Organisation,
Benutzenden und Problemen ab. Den Anfang bilden die Visionen, welche
realitätsnah Vorstellungen der Zukunft darstellen.

\setanf{V}
\begin{table}[htbp]
    \def\arraystretch{1.5}
    \centering
    \caption{Formale Visionen}
    \label{table:anf-vision}
    \begin{tabular}{m{0.08\textwidth}m{0.82\textwidth}}
        \uzlhline
        \anfrownumber & Veranstaltende sind besser in der Lage, ihre
        Veranstaltung zu planen, überblicken und auszuwerten.
        \\
        \anfrownumber & Veranstaltende sind besser in der Lage, mit
        Teilnehmenden zu kommunizieren.
        \\
        \anfrownumber & Teilnehmende werden motiviert und
        unterstützt, sich in der Veranstaltung zurechtzufinden.
        \\
        \uzlhline
    \end{tabular}
\end{table}

Auf Basis der Visionen werden Ziele formuliert, welche die Visionen
operationalisieren. Diese folgen dabei den standardisierten Regeln zur
Formulierung von Zielen \cite{Pohl2008}.

\setanf{Z}
\begin{table}[htbp]
    \def\arraystretch{1.5}
    \centering
    \caption{Formale Ziele}
    \label{table:anf-goals}
    \begin{tabular}{m{0.08\textwidth}m{0.82\textwidth}}
        \uzlhline
        \anfrownumber  & Veranstaltende sollen jederzeit in der Lage sein, Informationen
        zentral einzupflegen und diese zu verändern, um so effektiv auf
        Veränderungen einzugehen.
        \\
        \anfrownumber  & Veranstaltende und Teilnehmende sollen ab Beginn der Veranstaltung in der
        Lage sein, besser miteinander zu kommunizieren.                                            \\
        \textbf{/Z21/} & Veranstaltende sollen ab Beginn der Veranstaltung in der
        Lage sein, Teilnehmenden Benachrichtigungen zu senden, um diesen zeitnah
        mit wichtigen Informationen versorgen zu können.                                           \\
        \textbf{/Z22/} & Veranstaltende sollen ab Beginn der Veranstaltung in der
        Lage sein, Teilnehmende jederzeit um Feedback zu fragen, um Probleme
        frühzeitig zu erkennen.
        \\
        \textbf{/Z23/} & Teilnehmende sollen ab Beginn der Veranstaltung in der
        Lage sein, Veranstaltende unkompliziert und jederzeit zu kontaktieren.
        \\
        \anfrownumber  & Teilnehmende sollen ab Beginn der Veranstaltung in der
        Lage sein, sich über die Aktivitäten der Veranstaltung zu informieren,
        um die Navigation der Veranstaltung zu erleichtern.                                        \\
        \anfrownumber  & Teilnehmende sollen ab Beginn der Veranstaltung in der
        Lage sein, bei Schwierigkeiten schnell an hilfreiche Informationen zu
        gelangen.                                                                                  \\
        \anfrownumber  & Teilnehmende sollen ab Beginn der Veranstaltung in der
        Lage sein, Abzeichen erlangen zu können, um die aktive Teilnahme
        anzuregen.                                                                                 \\
        \anfrownumber  & Das System soll Informationen zugänglich präsentieren.
        \\
        \anfrownumber  & Veranstaltende sollen in der Lage sein, das Design des
        Systems auf ihre Veranstaltung anzupassen.
        \\
        \anfrownumber  & Veranstaltende sollen jederzeit in der Lage sein, vom
        System gesammelte Daten übersichtlich und strukturiert einzusehen.                         \\
        \uzlhline
    \end{tabular}
\end{table}

\newpage

\subsection{Rahmenbedingungen}

Die Rahmenbedingungen legen organisatorische und technische Restriktionen für
das System oder den Entwicklungsprozess fest \cite{Balzert2009}. Die Bedingungen
wurden aus der Benutzer und Kontextanalyse abgeleitet.

\setanf{R}
\begin{table}[htbp]
    \def\arraystretch{1.5}
    \centering
    \caption{Formale Rahmenbedingungen}
    \label{table:anf-circumstances}
    \begin{tabular}{m{0.08\textwidth}m{0.82\textwidth}}
        \uzlhline
        \anfrownumber & Das System ist eine Web-Anwendung.
        \\
        \anfrownumber & Die Zielgruppen sind Personen, welche Veranstaltungen
        organisieren und Teilnehmende dieser Veranstaltungen. Die
        Nutzungsgruppen wurden in \autoref{sec:analysis-user} definiert.
        \\
        \anfrownumber & Das System wird von Veranstaltenden in einem
        Arbeitsplatzsystemkontext genutzt und von Teilnehmenden vorwiegend in
        einem mobilen Kontext genutzt.                                        \\
        \anfrownumber & Das System wird über die Länge der Veranstaltung im
        Dauerbetrieb laufen. Das System soll auf Wunsch auch nach der
        Veranstaltung noch für beliebige Zeit erreichbar sein.                \\
        \anfrownumber & Das System muss unbeaufsichtigt zuverlässig lauffähig
        sein.                                                                 \\
        \anfrownumber & Auf den Zielmaschinen verwendete Software:
        \newline
        Client:
        Moderne marktführende Webbrowser (Chrome, Firefox, Edge, Safari)
        \\
        \uzlhline
    \end{tabular}
\end{table}

\subsection{Kontext und Überblick}

Jedes System ist in einer technischen Umgebung eingebettet \cite{Balzert2009}.
Im Folgenden wird Bezug auf das aktuelle System genommen.

\setanf{K}
\begin{table}[htbp]
    \def\arraystretch{1.5}
    \centering
    \caption{Formaler Kontext}
    \label{table:anf-context}
    \begin{tabular}{m{0.08\textwidth}m{0.82\textwidth}}
        \uzlhline
        \anfrownumber & Das aktuelle System ist eine Web-Anwendung, welche die
        Grundlage des neuen Systems darstellt.
        \\
        \uzlhline
    \end{tabular}
\end{table}

\subsection{Funktionale Anforderungen}

Die funktionalen Anforderungen halten die Kernfunktionalitäten des Systems fest
\cite{Balzert2009}. Diese ergeben sich aus allen Teilanalysen und den
festgelegten Zielen. Um eine eindeutige Semantik bei natürlichsprachlichen
Anforderungen zu gewährleisten, wird eine Anforderungsschablone (s. \autoref{fig:anf-schablone}) verwendet,
welche die Anforderungen vereinheitlicht \cite{Balzert2009}.

\begin{figure}[htpb]
    \renewcommand\baselinestretch{1}
    \centering
    \tikzset{
        textbox/.style={
                text width=4cm,
                text centered,
                align=center
            },
        arrow/.style={
                ->
            }
    }
    \tikz [thesis box shapes, baseline, anchor=base]{
        \node [block] (1) [textbox] at (-3, 8) {Den menschenzentrierten
            Gestaltungsprozess planen};
        \node [block] (2) [textbox] at (0, 3.75) {Den Nutzungskontext verstehen und
            beschreiben};
        \node [block] (3) [textbox] at (4.5, 0) {Die Nutzungsanforderungen
            spezifizieren};
        \node [block] (4) [textbox] at (0, -3) {Gestaltungslösungen entwickeln,
            die die Nutzungsanforderungen erfüllen};
        \node [block] (5) [textbox] at (-4.5, 0.25) {Gestaltungslösungen aus der
            Benutzerperspektive evaluieren};
        \node [block] (6) [textbox] at (-6, 4.5) {Gestaltungslösung
            erfüllt die Nutzungsanforderungen};
        \draw[arrow] (1.south) to [out=270, in=90] (2.north);
        \draw[arrow, bend left=45] (2.east) to (3.north);
        \draw[arrow, bend left=45] (3.south) to (4.east);
        \draw[arrow, bend left=45] (4.west) to (5.south);
        \draw[arrow] (5.west) to [out=180, in=270] (6.south);
        \draw[arrow, dashed, bend left=45] (5.north) to (2.west);
        \draw[arrow, dashed, bend left=35] (5.north) to (3.north);
        \draw[arrow, dashed, bend left=75] (5.north) to (4.north);
        \node [fill=white, fill opacity=0.7, text width=4cm, text opacity=1] (7) at (-3.5, 2) {Iteration, soweit
            Evaluationsergebnisse Bedarf hierfür aufzeigen};
    }
    \caption{Anforderungsschablone \cite{Balzert2009}}
    \label{fig:anf-schablone}
\end{figure}

\setanf{F}
\begin{table}[htbp]
    \def\arraystretch{1.5}
    \centering
    \caption{Formale funktionale Anforderungen}
    \label{table:anf-funk}
    \begin{tabular}{m{0.08\textwidth}m{0.82\textwidth}}
        \uzlhline
        \anfrownumber  & Das System \textit{muss} Veranstaltenden die
        Möglichkeit bieten, veranstaltungsrelevante Informationen einzutragen
        und jederzeit zu verändern (\anfref{Z}{10}).                                   \\
        \textbf{/F11/} & Das System \textit{muss} Veranstaltenden die
        Möglichkeit bieten, Stationen einzutragen und jederzeit zu verändern.          \\
        \textbf{/F12/} & Das System \textit{muss} Veranstaltenden die
        Möglichkeit bieten, Abzeichen einzutragen und jederzeit zu verändern.          \\
        \textbf{/F13/} & Das System \textit{muss} Veranstaltenden die
        Möglichkeit bieten, Hilfe (FAQ) jederzeit einzutragen oder zu verändern.
        \\
        \textbf{/F14/} & Das System \textit{muss} Veranstaltenden die
        Möglichkeit bieten, einführende Folien (Intro) jederzeit einzutragen
        oder zu verändern.
        \\
        \anfrownumber  & Das System \textit{muss} Veranstaltenden die Möglichkeit
        bieten, Statistiken zu den veranstaltungsbezogenen Aktivitäten
        einzusehen (\anfref{Z}{80}).                                                   \\
        \anfrownumber  & Das System \textit{muss} Teilnehmenden ab Beginn der
        Veranstaltung die Möglichkeit bieten, jederzeit Informationen
        einzusehen (\anfref{Z}{40}).                                                   \\
        \anfrownumber  & Das System \textit{muss} Veranstaltenden die
        Möglichkeit bieten, Teilnehmenden jederzeit Benachrichtigungen zu
        senden (s. \autoref{sec:analysis-problems-orga}: Organisation, \anfref{Z}{21}).
        \\
        \anfrownumber  & Das System \textit{soll} Veranstaltenden die
        Möglichkeit bieten, Teilnehmende jederzeit um Feedback bitten (s. \autoref{sec:analysis-problems-orga}: Organisation, \anfref{Z}{22}).
        \\
        \anfrownumber  & Das System \textit{soll} Veranstaltenden die
        Möglichkeit bieten, die Rahmendaten der Veranstaltung jederzeit
        einzutragen oder zu ändern (\anfref{Z}{10}).
        \\
        \anfrownumber  & Das System \textit{sollte in Zukunft} fähig sein,
        Informationen in verschiedenen Sprachen darzustellen (\anfref{Z}{70}).
        \\
        \anfrownumber  & Das System \textit{sollte in Zukunft} Teilnehmenden die
        Möglichkeit bieten, Veranstaltende jederzeit zu kontaktieren (\anfref{Z}{23}). \\
        \anfrownumber  & Das System \textit{sollte in Zukunft} umfassende
        Möglichkeiten zur Anpassung der Signalfarbe an die eigene Veranstaltung
        zulassen (\anfref{Z}{70}).
        \\
        \uzlhline
    \end{tabular}
\end{table}

\newpage

\subsection{Qualitätsanforderungen}

Abschließend werden Qualitätsanforderungen (auch nicht-funktionale
Anforderungen) festhalten. Diese legen qualitative oder quantitative
Eigenschaften des Systems fest \cite{Balzert2009}. Auch hier wird sich stark an
der Anforderungsschablone (s. \autoref{fig:anf-schablone}) orientiert.

\setanf{Q}
\begin{table}[htbp]
    \def\arraystretch{1.5}
    \centering
    \caption{Formale qualitative und quantitative Anforderungen}
    \label{table:anf-quali}
    \begin{tabular}{m{0.08\textwidth}m{0.82\textwidth}}
        \uzlhline
        \anfrownumber & Das System \textit{muss} den Grundsätzen der DIN EN ISO
        9241-110:2019-09 (Ergonomie der Mensch-System-Interaktion - Teil 110:
        Interaktionsprinzipien) folgen (\cite{iso-9241-210}).                                                                           \\
        \anfrownumber & Das System \textit{muss} die definierten Nutzungsklassen
        aus \autoref{sec:analysis-user} (Veranstaltende und Teilnehmende)
        unterscheiden und die dazugehörigen Zugriffsrechte sicherstellen.
        \\
        \anfrownumber & Das System \textit{muss} zuverlässig und ohne Störung im Dauerbetrieb laufen.
        \\
        \anfrownumber & Das System \textit{soll} beim Zugriff über das Internet eine gesicherte Übertragung (z. B. HTTPS) er-möglichen.
        \\
        \anfrownumber & Das System \textit{soll} alle Benutzerinteraktionen in unter fünf Sekunden ausführen.
        \\
        \uzlhline
    \end{tabular}
\end{table}

% Während der Veranstaltung Fokus Überwachung und Sicherstellung der Aktivitäten
% -> Die Stationsdashboards mit Nachrichten, Informationen und Übersicht zur
% Überwachung

% Mobiler Einsatz -> Betriebssysteme für Handys
%   - iOS vs Android https://gs.statcounter.com/os-market-share/mobile/europe/
%   - PWA / Web-App

% Stabilität und Zuverlässigkeit *sehr* wichtig
%   - Ausfall kann gesamtes Event lahmlegen
%   - Bei hoher Teilnehmeranzahl hohe Last

% Warum inclusive Design
%   - Menschen aller gesellschaftlichen Schichten nehmen an Veranstaltungen teil
%   - Jedem sollte die Chance geboten werden teilzunehmen

% Gruppengefühl, Soziale Interaktion -> Gruppenfunktion, Reaktionen

% Bessere Kommunikation -> Stationsdashboard