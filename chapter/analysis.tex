\chapter{Analyse}

% !===Veranstaltende===!

% Phasen der Veranstaltung
%   - Wo wird für Veranstalter angesetzt?
%   - Wie werden die Prozesse unterstützt?
%   - Was ist der Mehrwert?

% Herkömmliche Planung nur im Voraus
%   - Durch App mehr ad hoc handeln
%       - Push
%       - Feedback
%       - Reaktionen

% Wann setzt die App an?
%   - Werbephase?
%       - "Innovations"-Boost?
%   - Durchführung (duh)

% Achsen der Zugänglichkeit
%   - Körperliche Einschränkungen
%       - Blindheit
%       - Motorische Einschränkungen
%   - technische Barriere
%       - Geräte (iOS/Android)
%       - Affinität, besonders ältere Generationen
%       - Leistung der Geräte

% !===Teilnehmende===!

% Führung von Nutzern (Roter Pfaden)
%   - Gamification

% Verschiedene Szenarien
%   - Gezieltes öffnen
%   - Darüber stolpern

% Vergleichbare Projekte

% !===EMI-App===!

% Vorhandene Evaluationen und Erarbeitung der EMI-Award-App
%   - Ausweitung der Nutzenden
%   - Nutungskontext Veränderungen

% Marker-Problem
%   - Ausgedruckt
%       - nicht mehr (ohne großen Aufwand) änderbar
%       - Vandalismus

% !===Interviews===!

% Veranstaltende
%   - Erfassung des organisatorischen Ablaufs
%   - Feststellen der vorhandenen Probleme
%   - Lösungsansätze erkunden
%   - Evaluationsgrundlage erfahren
%   - Kommunikation erfassen

% Teilnehmende
%   - Bisherige Erfahrungen mit digitaler Anreicherung
%       - Positives & Negatives
%       - Bzw. überhaupt
%   - Hilfreiche Unterstützung erarbeiten
%   - Priorität des sozialen Austauschs
%       - Verschiedene Arten von Veranstaltungen
%   - Bevorzugte Informationsvermittlung (multi-modal)

% User Journey

% !===Erarbeitete Anforderungen / Funktionalitäten===!

Dieses Kapitel behandelt die Grundlagen, die für die Entwicklung des Frameworks
bedeutend sind. Es wurden vergleichbare Projekte, sowie grundlegende Konzepte
für Veranstaltungen gesucht. Gleichzeitig wurden die Ergebnisse der Evaluation
der EMI-Award-App gesichtet, um die dort gewonnen Erkenntnisse zur
Implementierung von Web-Apps sowie deren Einsatz zu erhalten. Daraufhin wurden
Interviews mit teilnehmenden und veranstaltenden Personen durchgeführt.

% Die Durchführung mit veranstaltenden Personen erfasste relevante
% Veranstaltungstypen, sowie wertvolle Funktionen aus organisatorischer Sicht.
% Hingegen erfassten die Interviews mit teilnehmenden Personen Erfahrungen mit
% digital unterstützen Veranstaltungen und hilfreiche vorhandene und gewünschte
% Funktionen.

\section{Nutzende}

\section{Nutzungskontext}

\section{Implikation für Konzeption}

% Stabilität und Zuverlässigkeit *sehr* wichtig
%   - Ausfall kann gesamtes Event lahmlegen
%   - Bei hoher Teilnehmeranzahl hohe Last

% Warum inclusive Design