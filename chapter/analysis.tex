\chapter{Analyse}

Dieses Kapitel behandelt die Analyse der Organisation und Teilnahme an
Veranstaltungen, sowie des Nutzungskontextes und der Nutzenden. Es wird ein
besonderer Fokus auf die verschiedenen Phasen einer Veranstaltung gelegt. Die
Umsetzung, Durchführung und Schließung werden näher analysiert. Unterstützt wird
die Analyse durch Interviews mit Personen, welche Erfahrung in der Organisation
von Veranstaltungen besitzen, sowie Interviews aus Sicht der Teilnehmenden.

\section{Datenquellen}

Im Rahmen der Analyse wurden verschiedene Datenquellen genutzt. Eine
wissenschaftliche Literaturrecherche wurde zwischen dem 26.07.2021 und
11.08.2021 durchgeführt. Verschiedene vergleichbare Projekte wurden im Rahmen
einer Internetrecherche ermittelt. Die Ausarbeitung und Evaluation der
EMI-Award-App wurde als Quelle genutzt. Zudem wurden Interviews mit
Veranstaltenden und Teilnehmenden durchgeführt. Die Interviews fanden zwischen
dem 5.09.2021 und 17.09.2021 statt. \\
Zur Literaturrecherche wurden die digitalen Bibliotheken ACM Digital Library
(\url{https://dl.acm.org/}) und Google Scholar
(\url{https://scholar.google.de/}) genutzt. Es wurden Begriffe aus Bereichen
Event Management (\emph{Event Management, Event Evaluation, Event Experience,
Event Tracking}) und UX (\emph{Attention Economy, Attention Management,
Information Overload, Digital Burnout, Overchoice, Web Animation, Animation UX,
UX Micro Interaction, Animation Usability}) zur Suche verwendet. \\
Die Interviews mit Veranstaltenden und Teilnehmenden wurden qualitativ und
semistrukturiert, mit Hilfe eines dafür entworfenen Interviewleitfadens (siehe
Anhang A), durchgeführt. Bei den Veranstaltenden handelt es sich um Personen mit
praktischer Erfahrung in der Organisation von Veranstaltungen. In Tabelle
\ref{table:ver-soz} und Tabelle \ref{table:teil-soz} sind Details zu den
befragten Personen aufgelistet.

\begin{table}[htpb]
    \def\arraystretch{1.25}
    \caption{Soziodemografische Daten der Veranstaltenden}
    \label{table:ver-soz}
    \centering
    \begin{tabular}{lll}
        \uzlhline
        \uzlemph{ID} & \uzlemph{Geschlecht} & \uzlemph{Vorerfahrung} \\
        \uzlhline V1 & w                    & Verschiedene Veranstaltungen,
        verteilt über 11 Jahre \\
        V2           & m                    & Mehrere große Veranstaltung (>2000
        Teilnehmende) \\
        \uzlhline
    \end{tabular}
\end{table}

\begin{table}[htpb]
    \def\arraystretch{1.25}
    \caption{Soziodemografische Daten der Teilnehmenden}
    \label{table:teil-soz}
    \centering
    \begin{tabular}{llll}
        \uzlhline
        \uzlemph{ID} & \uzlemph{Geschlecht} & \uzlemph{Alter} &
        \uzlemph{Tätigkeit} \\
        \uzlhline T1 & w                    & 18-25 J.        & Student:in \\
        T2           & m                    & 18-25 J.        & Student:in \\
        T3           & w                    & 18-25 J.        & Student:in \\
        T4           & m                    & 18-25 J.        & Student:in \\
        T5           & w                    & 40-59 J.        & Berufstätig \\
        \uzlhline
    \end{tabular}
\end{table}


\section{Definition: Veranstaltung} \label{sec:analysis-def}

Bevor die verschiedenen Aufgaben, Probleme und Benutzergruppen analysiert werden
können, müssen Umfang und Typ der Veranstaltung festgelegt werden. In der
Literatur sind viele verschiedene Definitionen zu Veranstaltungen zu finden,
welche sich in Umfang und Inhalt stark unterscheiden. Dies ist der Flexibilität
der räumlichen und zeitlichen Dimension von Veranstaltungen, sowie
Teilnehmeranzahl und Sektor geschuldet. Getz definiert Veranstaltungen
beispielsweise als zeitliche Phänomene. Bei geplanten Veranstaltungen wird das
Veranstaltungsprogramm oder der Zeitplan im Allgemeinen detailliert geplant und
im Voraus gut bekannt gemacht. Geplante Veranstaltungen sind in der Regel auch
auf bestimmte Orte beschränkt, obwohl es sich dabei um eine bestimmte
Einrichtung, eine sehr große Freifläche oder viele Orte handeln kann
\cite{Getz2007}. Bladen et al. definieren Veranstaltungen als zeitlich begrenzte
und zweckgebundene Zusammenkünfte von Menschen definiert \cite{Bladen2017}.

Im Rahmen dieser Arbeit werden Veranstaltungen betrachtet, welche in ihrer
zeitlichen Dimension uneingeschränkt sind, jedoch räumlich verteilt angesetzt
sind. Konkret können die betrachteten Veranstaltungen wenige Stunden, Wochen
oder ohne festgelegtes Ende stattfinden. Zudem verteilen sich die Aktivitäten
der Veranstaltung, sodass ein signifikanter Abstand zwischen ihnen vorliegt.
Dies kann mindestens die Aufteilung auf verschiedene Räumlichkeiten sein, bis
hin zur Verteilung über verschiedene Stadtteile.


\section{Organisationsanalyse} \label{sec:analysis-org}

% Masterarbeit_Holtz  Aufgabenanalyse!!!

Die Organisation von Veranstaltungen lässt sich in 5 Phasen Gliedern:
Initiation, Planung, Umsetzung, Durchführung und Schließung \cite{Silvers2013}.
Die Phasen sind wie folgt charakterisiert:

\begin{enumerate}
    \setlength{\itemsep}{1em}
    \item \textbf{Initiation} \\
          Es wird geforscht, um ein Konzept zu erstellen und zu validieren.
          Umfang und Kontext sowie Ziele und Aufgaben werden festgesetzt.
    \item \textbf{Planung} \\
          Anforderungen und Spezifikationen werden festgehalten. Hierzu zählen
          die stattfindenden Aktivitäten, sowie die Art der Organisation und
          erforderliche Ressourcen.
    \item \textbf{Umsetzung} \\
          Alle Waren und Dienstleistungen werden in Auftrag gegeben und
          koordiniert. Der Fokus liegt unter anderem auf der Überwachung und
          Überprüfung des Umfangs und Zeitplans, sowie der Kosten und Qualität.
    \item \textbf{Durchführung} \\
          Die veranstaltungsbezogenen Aktivitäten werden aufgenommen. Ab Beginn
          dieser Phase ist der Handlungsrahmen stark eingeschränkt, wodurch der
          Fokus auf der Überwachung der Veranstaltung liegt.
    \item \textbf{Schließung} \\
          Nach Abschluss der Veranstaltung stehen 4 wesentliche Punkte an:
          Feedback, Datensammlung, Bewertung und Wissenstransfer. Das Ziel ist
          die Dokumentation der Erkenntnisse für weitere Veranstaltungen.
\end{enumerate}

Im Folgenden werden die Umsetzung, Durchführung und Schließung näher betrachtet.

\subsection{Umsetzung} \label{sec:analysis-org-umsetzung}

Umsetzung

\subsection{Durchführung} \label{sec:analysis-org-durchfuehrung}

Mit Beginn der Durchführung ändert sich die Dynamik der Veranstaltung bedeutend.
Aufgrund der Anwesenheit von Teilnehmenden sind tiefgreifende Änderungen nun
nicht mehr möglich und beschränken sich auf die Behebung von kleinen Problemen.
Hingegen wird die wichtigste Aufgabe die Überwachung der Veranstaltung. Unter
besonderer Beobachtung stehen logistische Tätigkeiten, sowie unerwartet
auftretende Probleme. Diese können u. a. durch Teilnehmende, stattfindende
Aktivitäten oder die Umgebung der Veranstaltung ausgelöst werden
\cite{Bladen2017}. Während Probleme, welche die Durchführung der Veranstaltung
direkt beeinträchtigen könnten, streng beobachtet und behoben werden, ist die
Interaktion und Erfahrung der Teilnehmenden zweitrangig. Abgesehen von
kritischem Feedback, wird die Erfahrung erst während der Schließung erfasst und
ausgewertet.

\subsection{Schließung} \label{sec:analysis-org-schliessung}

% Bedeutung Evaluation aus Finischen Case S.57-59 Mischung Quantitativ &
% Qualitativ Evaluation S.61

Des Weiteren werden Evaluationen nur nach Abschluss der Veranstaltung
durchgeführt. Hierbei werden teilweise nur persönliche Erfahrungen der
Organisierenden und Helfenden zusammengetragen und ausgewertet. Dies ist der
mangelnden Datenerhebung während der Veranstaltung zuzurechnen.


\section{Benutzeranalyse} \label{sec:analysis-user}

In diesem Abschnitt werden die Benutzergruppen des Frameworks festgelegt und
näher vorgestellt. Zu den Benutzergruppen gehören Veranstaltende, sowie
Teilnehmende einer Veranstaltung.

\subsection{Veranstaltende}

% !===Veranstaltende===!

% Phasen der Veranstaltung
%   - Wo wird für Veranstalter angesetzt?
%   - Wie werden die Prozesse unterstützt?
%   - Was ist der Mehrwert?

% Herkömmliche Planung nur im Voraus
%   - Durch App mehr ad hoc handeln
%       - Push
%       - Feedback
%       - Reaktionen

% Wann setzt die App an?
%   - Werbephase?
%       - "Innovations"-Boost?
%   - Durchführung (duh)

Inzwischen existieren weltweit tausende Einrichtungen, welche formale
Qualifikationen und Ausbildungen im Veranstaltungsmanagement anbieten. Jedoch
sind diese Qualifikationen nicht einheitlich festgelegt, was zu
unterschiedlichen Schwerpunkten, Umfang, Vermittlungsart und letztendlich
erworbenen Qualifikationen führt \cite{Bladen2017}. In Deutschland gibt es den
anerkannten Ausbildungsberuf des/der Veranstaltungs\-kaufmann/-frau. Die
Aufgaben sind hierbei die Konzipierung und Organisation des kaufmännischen
Aspektes von Veranstaltungen \cite{Kultusministerkonferenz2001}. Die Aufgaben
eines/r Event Manager/in umfassen zusätzlich die allgemeine Organisation und
Aufgaben im Marketing \cite{BundesagenturfurArbeit2021}. Jedoch ist die
Berufsbezeichnung „Event Manager/in“ rechtlich nicht geschützt. Zudem bedarf es
in Deutschland keiner formalen Qualifikation, um eine Veranstaltung beliebiger
Größe zu organisieren.

\subsection{Teilnehmende}

Die Teilnehmenden einer Veranstaltung unterscheiden sich stark in ihren
soziodemografischen Daten je nach Typ der Veranstaltung.

Da die Literatur zu Event Management ihren Fokus auf die Organisation legt, sind
nur wenig Informationen zur Sicht der Teilnehmenden und ihren Hürden vorhanden.
Als Messwerte für eine erfolgreiche Veranstaltung werden meist
geschäftsrelevante Werte wie z. B. verkaufte Tickets, Teilnehmerzahlen, Umsatz
oder öffentliche Aufmerksamkeit verwendet, welche wenig über die Erfahrung der
Teilnehmenden aussagen.

% !===Teilnehmende===!

% !Alter, Soziale Schicht, Technik Affinität

% Führung von Nutzern (Roter Pfaden)
%   - Gamification

% Verschiedene Szenarien
%   - Gezieltes öffnen
%   - Darüber stolpern


\section{Kontextanalyse} \label{sec:analysis-context}

In diesem Abschnitt werden der zeitliche und räumliche Kontext untersucht. Aus
der Beschreibung in \autoref{sec:goals} geht der Fokus auf die Unterstützung von
Veranstaltenden und Teilnehmenden hervor. Die Unterteilung der Benutzergruppen,
wie in \autoref{sec:analysis-user} beschrieben, ist zu beachten. Hieraus ergeben
sich unterschiedliche räumliche und zeitliche Kontexte für Veranstaltende und
Teilnehmende.

Die Unterstützung der Veranstaltenden findet in verschiedene Phasen der
Organisation statt. Der zeitliche Kontext kann mit Blick auf
\autoref{sec:analysis-org} in vor, während und nach einer Veranstaltung
unterteilt werden. Vor der Durchführung einer Veranstaltung ist insbesondere die
Vorbereitung von Aktivitäten und Absicherung von Dienstleistungen oder Waren
wichtig. Während der Veranstaltung hingegen findet ein starker Fokuswechsel auf
die Beobachtung und schnelle Behebung von Problemen statt. Nach einer
Veranstaltung sind die Datensammlung und Auswertung, sowie der Wissenstransfer
von Bedeutung. \\
Auch beschränken sich die betrachteten Veranstaltungen während der Durchführung
auf den in \autoref{sec:analysis-def} beschriebenen örtlichen Bereich, woraus
sich ein weites gehend unbeschränkter räumlicher Kontext ergibt. Vor und nach
der Veranstaltung werden komplexe Informationen eingetragen oder ausgewertet. Da
die Bildschirmgröße einen großen Einfluss auf die effiziente Verarbeitung von
Informationen hat, wird hierfür von einem Arbeitsplatz mit großem Bildschirm
ausgegangen. Somit ergibt sich ein räumlich eingeschränkter Kontext vor und nach
der Veranstaltung.

Für Teilnehmende ist der Kontext ebenfalls in vor, während und nach der
Veranstaltung aufgeteilt. Jedoch unterscheiden sich die Kontexte in ihren
Inhalten stark.


% Während der Veranstaltung Fokus Überwachung und Sicherstellung der Aktivitäten
% -> Die Stationsdashboards mit Nachrichten, Informationen und Übersicht zur
% Überwachung

% Der Framework muss für Teilnehmende folglich mobil einsetzbar sein.

% Trennung Kontext Veranstalter - Teilnehmer
% - Zusammentreffen erst ab Durchführung
% - Unterschiede im Kontext

% Achsen der Zugänglichkeit
%   - Körperliche Einschränkungen
%       - Blindheit
%       - Motorische Einschränkungen
%   - technische Barriere
%       - Geräte (iOS/Android)
%       - Affinität, besonders ältere Generationen
%       - Leistung der Geräte

\section{Problemanalyse} \label{sec:analysis-problems}

Die Interviews der Veranstaltenden bestätigen, dass die Überwachung der
Aktivitäten nur auf der Makroebene erfolgt, da Interaktionen auf Mikroebene
schwer zu verfolgen sind. Je nach Veranstaltungstyp liegt der Fokus auf der
Sicherheit und Logistik der Veranstaltung. Evaluationen werden meist nach
Veranstaltungen durchgeführt, jedoch nicht während Veranstaltungen. In seltenen
Fällen wurde eine Evaluation vor der Veranstaltung durchgeführt, um die
Erwartungen der Teilnehmenden zu erfassen.

Einen weiteren Schwerpunkt stellt effiziente Kommunikation dar, welche zur
schnellen Behebung von Problemen benötigt wird. Alle Mitglieder des Teams
sollten ihre Umgebung im Blick haben und problematische Beobachtung unverzüglich
kommunizieren können. Aus den Interviews ergab sich, dass die Kommunikation
innerhalb des veranstaltenden Teams, aber auch besonders zu Teilnehmenden eine
Herausforderung darstellt.

% Nachfrage nach Event-Assistenten

% !===EMI-App===!

% Vorhandene Evaluationen und Erarbeitung der EMI-Award-App
%   - Ausweitung der Nutzenden
%   - Nutungskontext Veränderungen

% Marker-Problem
%   - Ausgedruckt
%       - nicht mehr (ohne großen Aufwand) änderbar
%       - Vandalismus

% !===Interviews===!

% Veranstaltende
%   - Erfassung des organisatorischen Ablaufs
%   - Feststellen der vorhandenen Probleme
%   - Lösungsansätze erkunden
%   - Evaluationsgrundlage erfahren
%   - Kommunikation erfassen

% Teilnehmende
%   - Bisherige Erfahrungen mit digitaler Anreicherung
%       - Positives & Negatives
%       - Bzw. überhaupt
%   - Hilfreiche Unterstützung erarbeiten
%   - Priorität des sozialen Austauschs
%       - Verschiedene Arten von Veranstaltungen
%   - Bevorzugte Informationsvermittlung (multi-modal)

% User Journey

% !===Erarbeitete Anforderungen / Funktionalitäten===!


% Die Durchführung mit veranstaltenden Personen erfasste relevante
% Veranstaltungstypen, sowie wertvolle Funktionen aus organisatorischer Sicht.
% Hingegen erfassten die Interviews mit teilnehmenden Personen Erfahrungen mit
% digital unterstützen Veranstaltungen und hilfreiche vorhandene und gewünschte
% Funktionen.

% - Alle gesellschaftlichen Schichten
% - Alle Altersklassen
% - Menschen mit körperlichen Einschränkungen


\section{Implikation für Konzeption} \label{sec:analysis-implic}

% Betriebssysteme für Handys
%   - iOS vs Android https://gs.statcounter.com/os-market-share/mobile/europe/

% Stabilität und Zuverlässigkeit *sehr* wichtig
%   - Ausfall kann gesamtes Event lahmlegen
%   - Bei hoher Teilnehmeranzahl hohe Last

% Warum inclusive Design
%   - Menschen aller gesellschaftlichen Schichten nehmen an Veranstaltungen teil
%   - Jedem sollte die Chance geboten werden teilzunehmen