\chapter{Zusammenfassung und Ausblick} \label{chapter:summary}

Abschließend werden die wichtigsten Ergebnisse der Arbeit zusammengefasst.
Außerdem wird auf die technischen Probleme der realisierten Funktionen während
der Nutzung eingegangen. Zum Abschluss wird ein Ausblick für die möglichen
Weiterentwicklungen des Frameworks gegeben.

\section{Zusammenfassung}

Im Rahmen dieser Arbeit wurde eine Erweiterung und Verallgemeinerung der
EMI-Award-App in Form eines flexiblen Event-App-Frameworks für ortsbasierte
Veranstaltungen konzipiert, entwickelt und evaluiert.
\\
Um ein Verständnis für den Veranstaltungskontext und die Herausforderungen der
Nutzergruppen zu erlangen, wurden eine Literaturrecherche und Interviews mit
Teilnehmenden und Veranstaltenden durchgeführt (vgl.
\autoref{chapter:analysis}). Aus den Interviews und der Literatur ergab sich,
dass beide Nutzergruppen in unterschiedlichen Veranstaltungphasen Unterstützung
bedürfen. Des Weiteren wurde das bestehende System, die EMI-Award-App,
analysiert, um den Stand der Technik zu beleuchten. Anschließend wurden die
gesammelten Erkenntnisse in Form von formalen Anforderungen festgehalten.

Auf Basis der formalen Anforderungen wurden die Funktionalitäten des Frameworks
konzipiert. Hierbei wurde auf den Funktionalitäten des bestehenden Systems
aufgebaut und weitere Funktionalitäten hinzugefügt, um die Anforderungen
abzudecken. Anschließend wurde das Interfacedesign des Dashboards und der
Web-App entwickelt. Weiterhin wurden auf Grundlage der Anforderungen die in
dieser Arbeit verwendeten Frameworks festgelegt und darauf aufbauend die
Systemarchitektur festgelegt.

Anschließend wurden die zuvor konzipierten Entwürfe des Systems realisiert.
Hierbei wurden die in der Konzeptionsphase festgelegten Frameworks
\textit{Vue.js} und \textit{Strapi} verwendet. Darauf wurde das implementierte
System anhand von Dialogbeispielen präsentiert.

Abschließend wurde das entwickelte System von Teilnehmenden und
Veranstaltenden evaluiert. Für die Seite der Veranstaltenden wurde erhoben,
inwieweit das System in der Organisation und Durchführung unterstützt. Ein
besonderes Merkmal wird hierbei auf die Überwachung der Veranstaltung gesetzt.
Durch die Teilnehmenden wurde erhoben, ob und wie sehr die implementierten
Funktionalitäten zur Orientierung während des Besuchs der Veranstaltung
unterstützen konnten.

\section{Offen Punkte}

In der Evaluation des Systems wurden unterschiedliche technische Problemen
festgestellt, welche in offen gebliebenen Punkten resultieren. Im Folgenden
werden die Probleme und die daraus resultierenden offenen Punkte näher
erläutert.

Die Kamerafunktion zum virtuellen Besuchen und Senden von Bildabzeichen
funktioniert bisher nicht zuverlässig. Bildabzeichen nutzen standardmäßig die
erstgefundene Kamera des Systems. Da diese meist die Frontkamera des Geräts ist,
erschwert dies die Aufnahme von Bildern für Teilnehmende. An dieser Stelle
sollte in Zukunft die Auswahl der Kamera gegeben werden, um Teilnehmende nicht
einzuschränken. Die Kamera zum virtuellen Besuchen hingegen lässt nach kurzer
Zeit (> 3 min) die Web-App abstürzen. Dies ist ein bekanntes Problem des
genutzten
Paketes\footnote{\url{https://github.com/gruhn/vue-qrcode-reader/issues/233}}
für die QR-Code-Reader Funktion. Da bisher keine vergleichsweisen Pakete
existieren, sollte in Zukunft auf Aktualisierungen des Pakets gewartet
werden.

Ein weiteres Problem besteht in der Zustellung von Benachrichtigungen. Diese
werden wie in \ssecref{ssec:impl-backend-push} beschrieben mit der Web-Push API
umgesetzt. Diese wird von Safari auf iOS noch nicht unterstützt, weshalb zur
Absicherung Web-Sockets genutzt werden. Jedoch ist somit bisher keine
Benachrichtigung außerhalb der Nutzung der App für iOS Nutzende möglich. Auch
hier sollte die Entwicklung der iOS Unterstützung geprüft werden und bei
erfolgter Unterstützung die Implementierung angepasst werden.

\section{Ausblick}

In der Evaluation wurden Proband:innen nach Funktionalitäten gefragt, welche sie
während der Nutzung vemisst hätten. Hierbei ist eine Aufzählung von potentiellen
weiteren Funktionalitäten entstanden (vgl. \ssecref{ssec:eval-t-results}). Um
den Beobachtungen der Proband:innen nachzukommen, sollten diese Vorschläge näher
analysiert werden.

Einige der potentiellen Funktionalitäten sind bereits in Form von
formalisierten Anforderungen aufgestellt worden (vgl.
\ssecref{ssec:analysis-anf-func}). Hierzu zählen die erweiterten
Kontaktmöglichkeiten zwischen Teilnehmenden und Veranstaltenden. Aufgrund der
Überschneidung, solle diesen Funktionalitäten Priorität gewährt werden.

Einige Funktionalitäten des zuvor bestehenden Systems (vgl.
\autoref{sec:analysis-old}) wurden bisher nicht auf diesen Framework übertragen.
Hierzu zählt u. a. die AR-Interkation während des virtuellen Besuchens sowie die
Bewertung der einzelnen Stationen. Die AR-Interaktion stellt im Kontext eines
allgemeinen Frameworks eine besondere Herausforderung dar, aufgrund des mit der
Konfiguration verbundenen Aufwands für Veranstaltende. Diese Funktionalitäten
könnten im Rahmen einer Folgearbeit erneut aufgegriffen werden.
