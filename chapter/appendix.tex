\setlength{\parskip}{2pt}

\chapter{Interviewleitfäden: Analyse} \label{appendix:interview}

\section{Teilnehmende}


\textbf{\large Einstieg}

\begin{enumerate}[noitemsep,topsep=0pt]
    \item Begrüßung und Danken für die Zeit
    \item Kurzer Umriss des Themas
    \item Ist die EMI-Award-App bekannt?
    \item Alter, Studiengang / Tätigkeit
    \item Kontaktmöglichkeit im Nachhinein
    \item Datenschutz
\end{enumerate}


\textbf{\large Einstiegsfrage}

\begin{enumerate}[noitemsep,topsep=0pt]
    \item {
        Welche von Ihnen besuchte Veranstaltung ist Ihr persönlicher Favorit?
        \begin{enumerate}[noitemsep,topsep=0pt]
            \item Berücksichtigen: Covid, lange keine Präsenz Veranstaltungen
            \item Details zur Veranstaltung erfragen, falls nicht bekannt
        \end{enumerate}
        }
    \item Was macht diese Veranstaltung zu ihrem persönlichen Favoriten?
\end{enumerate}


\textbf{\large Schlüsselfragen}

\textbf{Frage 1:} Wenn Sie sich etwas hilfreiches begleitend zur Veranstaltung wünschen könnten, was würde dies sein?

\textbf{Frage 2:} Welchen Technologien sind Sie auf Veranstaltungen schon begegnet? (Oder auch nicht)
\begin{itemize}[noitemsep,topsep=0pt]
    \item Hat Ihnen die Technologie auf der Veranstaltung geholfen?
    \item Wenn ja, wie? Sonst, warum nicht?
\end{itemize}

\textbf{Frage 3:} Wie wichtig ist Ihnen der soziale Austausch mit anderen Teilnehmenden während einer Veranstaltung?
\begin{itemize}[noitemsep,topsep=0pt]
    \item Wie verändert sich dies mit dem Typ der Veranstaltung?
\end{itemize}

\textbf{Frage 4:} Welchen (multi-/medialen) Weg bevorzugen Sie um Informationen aufzunehmen? (Videos schauen, Berichte/Artikel lesen, Podcasts hören, ...)
\begin{itemize}[noitemsep,topsep=0pt]
    \item Wie verändert sich dies in einem lehrreichen / unterhaltenden Setting?
\end{itemize}


\textbf{\large Abschluss}

\begin{enumerate}[noitemsep,topsep=0pt]
    \item Nochmals für die Zeit Danken
    \item Kontaktmöglichkeit im Nachhinein
    \item Verabschiedung
\end{enumerate}


\section{Veranstaltende}

\textbf{\large Einstieg}

\begin{enumerate}[noitemsep,topsep=0pt]
    \item Begrüßung und Danken für die Zeit
    \item Kurzer Umriss des Themas
    \item Vorerfahrung (Wie viele Veranstaltungen? Wie Groß?)
    \item Kontaktmöglichkeit im Nachhinein
    \item Datenschutz
\end{enumerate}

\textbf{\large Einstiegsfrage: Phase Organisation}

\begin{enumerate}[noitemsep,topsep=0pt]
    \item  Welche von Ihnen (mit)organisierte Veranstaltung ist Ihr persönlicher
          Favorit?
    \item {
          Beschreiben Sie grob den Ablauf bei der Planung der Veranstaltung
          \begin{enumerate}[noitemsep,topsep=0pt]
              \item Ob und wie wird Feedback von Teilnehmenden während Veranstaltung gesammelt?
              \item Ob und wie Kontakt zu Teilnehmenden während Veranstaltung?
              \item Ob und wie Kontakt zu Teilnehmenden nach Veranstaltung?
              \item Welche Priorität hat die Zugänglichkeit der Veranstaltung?
              \item Was wird für die Zugänglichkeit getan?
          \end{enumerate}
          }
\end{enumerate}


\textbf{\large Schlüsselfragen}

\textbf{Frage 1 (Organisation/Durchführung):} Wenn Sie sich etwas hilfreiches begleitend zur Veranstaltung wünschen könnten, was würde dies sein? Explizit Organisatorisch

\textbf{Frage 2 (Durchführung):} Was für Feedback ist während einer
Veranstaltung wichtig? (Worauf kann realistisch noch eingegangen werden?)
\begin{itemize}[noitemsep,topsep=0pt]
    \item Welche Daten werden benötigt in der Nachbereitung?
\end{itemize}

\textbf{Frage 3 (Durchführung/Nachbereitung):} Wenn die Möglichkeit bestände
Teilnehmer während einer Veranstaltung direkt / jeder Zeit anzusprechen, wofür
würden Sie das nutzen?
\begin{itemize}[noitemsep,topsep=0pt]
    \item Wofür nach einer Veranstaltung?
\end{itemize}

\textbf{\large Abschluss}

\begin{enumerate}[noitemsep,topsep=0pt]
    \item Nochmals für die Zeit Danken
    \item Kontaktmöglichkeit im Nachhinein
    \item Verabschiedung
\end{enumerate}

\chapter{Interviewleitfaden: Evaluation} \label{appendix:evaluation}

\textbf{\large Einstieg}

\begin{enumerate}[noitemsep,topsep=0pt]
    \item Begrüßung und Danken für die Zeit
    \item
        Gesamteindruck
        \begin{enumerate}[noitemsep,topsep=0pt]
            \item Was ist positiv aufgefallen?
            \item Was ist negativ aufgefallen?
        \end{enumerate}
        \begin{itemize}[noitemsep,topsep=0pt]
            \item[->] darauf dann wieder eingehen
        \end{itemize}
\end{enumerate}

\textbf{\large Details zum Ablauf }

\begin{enumerate}[noitemsep,topsep=0pt]
    \item  Wie lief das Erstellen der Veranstaltung?
    \begin{enumerate}[noitemsep,topsep=0pt]
        \item Eintragen der Informationen (Stationen, Abzeichen, Intro)
        \item Einstellen des Standorts
        \item optional aus Frage 2 übernehmen
    \end{enumerate}
    \item Wie lief das “Überwachen” der Veranstaltung?
    \begin{enumerate}[noitemsep,topsep=0pt]
              \item Einsehen der Statistiken
              \item Bewerten der Abzeichen
              \item Senden von Benachrichtigungen / Feedback
              \item Einstellen der Rahmendaten
              \item optional aus Frage 2 übernehmen
    \end{enumerate}
    \item Noch Ideen? Noch was gefehlt? Was vermisst du aktuell?
\end{enumerate}

\textbf{\large Abschluss}

\begin{enumerate}[noitemsep,topsep=0pt]
    \item Nochmals für die Zeit Danken
    \item Kontaktmöglichkeit im Nachhinein
    \item Verabschiedung
\end{enumerate}

\chapter{Digitale Medien}

Auf der beigefügten CD sind folgende Inhalte zu finden:
\begin{enumerate}[noitemsep,topsep=0pt]
    \item Aufnahmen der geführten Interviews (Verzeichnis \textit{/befragung})
    \item Detaillierte Ergebnisse der Evaluation (Verzeichnis \textit{/evaluation})
    \item Quellcode des entwickelten Systems (Verzeichnis \textit{/framework})
    \item PDF-Version dieser Arbeit
\end{enumerate}

