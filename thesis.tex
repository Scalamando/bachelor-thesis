\documentclass[german, version-2020-11]{uzl-thesis}

\UzLThesisSetup{
    Logo-Dateiname          = {res/imis-thesis-logo.pdf},
    Verfasst                = {am}{Institut für Multimediale und Interaktive Systeme},
    Titel auf Deutsch       = {
        Event-App Framework für ortsbezogene Veranstaltungen
    },
    Titel auf Englisch 	    = {
        Event-App Framework for location-based Events
    },
    Autor                   = {Raimund Canzler},
    Betreuerin              = {Univ.-Prof. Dr.-Ing. Nicole Jochems Dipl.-Inform.},
    Mit Unterstützung von   = {Torben Volkmann, M.Sc.},
    Bachelorarbeit,
    Studiengang       	    = {Medieninformatik},
    Datum                   = {20. Februar 2022},
    Abstract                = { Digital media have greatly facilitated access to
        events over the last 10 years. However, online-only events lack an
        important aspect of local events: face-to-face interaction. In the
        context of the EMI Award 2021, an app was implemented which, among other
        things, made it possible to view and rate the projects presented there.
        In the context of this bachelor thesis, an extension and generalisation
        of the EMI-Award app was designed and developed in the form of a
        flexible event app framework for location-based events. The framework
        supports event organisers in the organisation and implementation of
        events by allowing them to manage event-relevant information via a web
        interface. Participants access is facilitated in a clearly structured
        web app. There they can view and virtually visit stations, submit badges
        and view help entries. As a basis, the existing EMI-Award app was
        analysed. In addition, interviews with participants and organisers were
        conducted to understand the processes of organising the event and
        problems in the implementation and participation. The system was then
        designed and implemented. Finally, the system was used at three events
        and evaluated with participants and organizers. The evaluation showed a
        support during the event, both for the participants and for the
        organizers. for both participants and organizers.
        \linebreak\linebreak\linebreak
        {\large Keywords\linebreak}
        \vspace{10mm}
        Event-Management, Dashboard, Progressive Web App, Framework
    },
    Zusammenfassung         = { Die digitalen Medien haben den Zugang zu
        Veranstaltungen in den letzten 10 Jahren erheblich erleichtert.
        Allerdings fehlt bei reinen Online-Veranstaltungen ein wichtiger Aspekt
        lokaler Veranstaltungen: die persönliche Interaktion. Im Rahmen des EMI
        Award 2021 wurde eine App implementiert, die es u.a. ermöglichte, die
        dort präsentierten Projekte zu betrachten und zu bewerten. Im Rahmen
        dieser Bachelorarbeit wurde eine Erweiterung und Verallgemeinerung der
        EMI-Award-App in Form eines flexiblen Event-App-Frameworks für
        ortsbezogene Veranstaltungen konzipiert und entwickelt. Das Framework
        unterstützt Veranstaltende bei der Organisation und Durchführung von
        Veranstaltungen, indem es ihnen ermöglicht, veranstaltungsrelevante
        Informationen über ein Webinterface zu verwalten. Teilnehmenden wird der
        Zugang in einer übersichtlichen Web-App erleichtert. Dort können sie
        Stationen einsehen und virtuell besuchen, Badges abgeben und
        Hilfeeinträge einsehen. Als Grundlage wurde die bestehende EMI-Award-App
        analysiert. Darüber hinaus wurden Interviews mit Teilnehmenden und
        Veranstaltenden geführt, um die Abläufe bei der Organisation der
        Veranstaltung und Probleme bei der Umsetzung und Teilnahme zu verstehen.
        Anschließend wurde das System entworfen und implementiert. Schließlich
        wurde das System bei drei Veranstaltungen eingesetzt und mit
        Teilnehmenden und Veranstaltenden evaluiert. Die Evaluation ergab eine
        stark positiv empfundene Unterstützung während der Veranstaltung, sowohl
        für Teilnehmende als auch Veranstaltende. \linebreak\linebreak\linebreak
        {\large Schlüsselwörter\linebreak}
        \vspace{10mm}
        Event-Management, Dashboard, Progressive Web App, Framework
    },
    Alphabetische Bibliographie
}

\UzLStyle{alegrya modern design}

\addbibresource{sources.bib}


\DeclareBibliographyDriver{dindriver}{%
\usebibmacro{begentry}%
\ifboolexpr{ test {\iffieldundef{shorttitle}}}
{\usedriver{}{\thefield{entrytype}}}
{\printtext[bibhyperref]{\printfield{shorttitle}}}%
\usebibmacro{finentry}%
}

\DeclareCiteCommand{\citedin}[\mkbibparens]
{\usebibmacro{prenote}}
{\usedriver{}{dindriver}}
{\multicitedelim}
{\usebibmacro{postnote}}

\usepackage{tabularx}
\usepackage{threeparttable}
\usepackage{enumitem}
\usepackage{multirow}

%%%%%%%%%%%%
% GRAFIKEN %
%%%%%%%%%%%%

\graphicspath{ {./res/images/} }

\newcommand{\textgraphics}[2][height=.5cm]{$\vcenter{\hbox{\includegraphics[#1]{#2}}}$}


%%%%%%%%%%%%%%%%%
% ANFORDERUNGEN %
%%%%%%%%%%%%%%%%%

\usepackage{array}
\usepackage{etoolbox}
\usepackage{longtable}

\newcommand\anfletter{Q}

\newcommand\setanf[1]{\renewcommand\anfletter{#1}}

\newcounter{anfnumber}
\newcounter{anfsubnumber}
\newcommand\showanfnumber{
    \textbf{/\anfletter\arabic{anfnumber}\arabic{anfsubnumber}/}
    \def\@currentlabel{/\arabic{anfnumber}\arabic{anfsubnumber}/}
    \label{anf:\anfletter\arabic{anfnumber}\arabic{anfsubnumber}}
}

\newcommand\anfrow{\setcounter{anfsubnumber}{0}\refstepcounter{anfnumber}\showanfnumber}
\newcommand\anfsubrow{\refstepcounter{anfsubnumber}\showanfnumber}

\newcommand{\anfref}[1]{\hyperref[anf:#1]{/#1/}}

\preto\tabular{\setcounter{anfnumber}{0}}
\preto\tabular{\setcounter{anfsubnumber}{0}}

\preto\longtable{\setcounter{anfnumber}{0}}
\preto\longtable{\setcounter{anfsubnumber}{0}}

%%%%%%%
% REF %
%%%%%%%
\newcommand{\ssecref}[1]{\hyperref[{#1}]{\autoref*{#1}: „\nameref*{#1}“}}

%%%%%%%%
% ABBR %
%%%%%%%%

\DeclareAcronym{CMS}{
    short = CMS,
    long = Content-Management-System,
}
\DeclareAcronym{FAQ}{
    short = FAQ,
    long = „Frage \& Antwort“,
}

\DeclareAcronym{PWA}{
    short = PWA,
    long = progressive Web App,
}

\DeclareAcronym{AR}{
    short = AR,
    long = Augmented Reality
}

\DeclareAcronym{VR}{
    short = VR,
    long = Virtual Reality
}

\DeclareAcronym{XR}{
    short = XR,
    long = Extended Reality
}

%%%%%%%%%%%%
% DOKUMENT %
%%%%%%%%%%%%

\begin{document}

\setlength{\parindent}{0em}
\setlength{\parskip}{1em}

\chapter{Einleitung}

% TODO: Quellen überprüfen!

In den letzten 20 Jahren hat sich am Kern der Veranstaltungsorganisation wenig
verändert, jedoch bietet moderne Technik Veranstaltungen eine größere Bühne als
je zuvor. Digitale Medien haben in den letzten 10 Jahren den Zugang zu
Veranstaltungen enorm erleichtert \cite{Bladen2012}. Webinars,
Online-Konferenzen, -Training und -Workshops bieten Menschen quer über die Welt
verteilt die Möglichkeit, an Veranstaltungen teilzunehmen und dabei zu wählen,
was sie wann und wie sehen möchten. Jedoch fehlt reinen Online Veranstaltungen
ein wichtiger Aspekt von örtlich stattfindenden Veranstaltungen: die
Face-to-face Interaktion \cite{Bladen2012}. Soziale Interaktion ist ein wichtiges
psychologisches Grundbedürfnis \cite{Maslow1943}.

Anfang 2021 waren soziale Interaktionen durch die pandemische Lage nur sehr
eingeschränkt möglich. Zum EMI-Award 2021 wurde deshalb am Institut für
Multimediale und Interaktive Systeme (IMIS) der Universität zu Lübeck im Rahmen
eines Bachelor-Projekts die EMI-Award-App umgesetzt \cite{Canzler2021}. Beim
EMI-Award handelt es sich um eine jährliche Veranstaltung des Studiengangs
Medieninformatik, bei welcher „die kreativsten und gelungensten
Gruppenergebnisse aus der Erstsemester-Veranstaltung ‚Einführung in die
Medieninformatik‘ (EMI) ausgezeichnet [werden]“ \cite{UniversitatzuLubeck2021}.
Aufgrund der Kontaktbeschränkung konnten, im Gegensatz zu den vorherigen Jahren,
die Projekte der Studenten nicht im Foyer des Audimax an der Universität zu
Lübeck vorgestellt werden. Die App ermöglichte das Ansehen und Bewerten der im
gesamten Raum Lübeck verteilten Projekte. Innerhalb von 2 Wochen wurde die
Web-App von über 300 Personen genutzt. Teilnehmende erhielten die Möglichkeit,
die Projekte in ihrer eigenen Geschwindigkeit zu erkunden. Zudem ging aus
informellen Erfahrungsberichten hervor, dass Teilnehmende an Projektstandorten
öfter aufeinandertrafen.

% App, welche Veranstaltungen digital unterstüzt
%   - Nutzer einen roten Pfaden bieten
%       - Einführende Informationen zur Veranstaltung
%       - Informieren zu Standorten, Neuigkeiten
%       - Fortschrittsüberblick durch virtuelle Besuche / Einchecken
%       - Gruppen-Funktion
%       - soziale Interaktion durch Reaktionen
%       - Beteiligung durch Gamification
%       - Hilfe anbieten, häufige Probleme lösen
%   - Veranstalter die Informationsverteilung und -aufnahme stark vereinfachen
%       - Überblicken der Veranstaltung
%       - Spontan Feedback sammeln
%       - Jederzeitiges Eingreifen durch Benachrichtigungen
%       - Häufige Probleme dank FAQ abfangen
%       - Datensammlung für Evaluationszwecke
%   - Einfache Nutzung dank Web-App
%   - Keine Registration

\section{Ziele dieser Arbeit} \label{sec:goals}

Im Rahmen dieser Bachelorarbeit wurde eine Erweiterung und Verallgemeinerung der
EMI-Award-App in Form eines flexiblen AR-Event-App-Frameworks für ortsbasierte
Veranstaltungen konzipiert und entwickelt. Dabei wurde die Organisation von
Veranstaltungen und Teilnahme näher analysiert. Zudem wurden die Vor- und
Nachteile der bestehenden Anwendung einbezogen. Der Framework unterstützt
Veranstaltende ab der „Umsetzung“ und Teilnehmende ab der „Durchführung“ einer Veranstaltung.

Veranstaltende können über ein Online-Dashboard ihre Veranstaltung verwalten.
Dies umfasst das Anlegen von verschiedenen Standorten, Abzeichen und
Hilfseinträge. Des Weiteren sollen Push-Nachrichten sowie Feedbackanfragen an
alle Teilnehmenden abgeschickt werden können. Zusätzlich ermöglicht die Web-App
Daten zu besuchten Standorten, erhaltenen Abzeichen und Feedback gesammelt
darzustellen. Diese Daten können anschließend zur Auswertung der Veranstaltung
genutzt werden.

Teilnehmenden wird durch die übersichtliche Darstellung wichtiger Informationen
zur Veranstaltung ein roter Faden geboten. Dies geschieht mithilfe einer
Anzeige der Standorte, sowohl auf einer interaktiven Karte, als auch in
Listenform. Zudem können weitere Informationen und multimediale Inhalte zu den
einzelnen Standorten eingesehen werden. Das virtuelle Besuchen von Standorten
per QR-Code und Abzeichen sollen Teilnehmende zur Ausführung von bestimmten
gewünschten Aktionen lenken.

\section{Aufbau der Arbeit}

% Herczeg

Das Vorgehen dieser Arbeit basiert auf dem menschenzentrierten Gestaltungsprozess nach DIN
EN ISO 9241-210 (s. \autoref{fig:din-hcd}).

Im zweiten <blablabla...>

\begin{figure}[htpb]
    \renewcommand\baselinestretch{1}
    \centering
    \tikzset{
        textbox/.style={
                text width=4cm,
                text centered,
                align=center
            },
        arrow/.style={
                ->
            }
    }
    \tikz [thesis box shapes, baseline, anchor=base]{
        \node [block] (1) [textbox] at (-3, 8) {Den menschenzentrierten
            Gestaltungsprozess planen};
        \node [block] (2) [textbox] at (0, 3.75) {Den Nutzungskontext verstehen und
            beschreiben};
        \node [block] (3) [textbox] at (4.5, 0) {Die Nutzungsanforderungen
            spezifizieren};
        \node [block] (4) [textbox] at (0, -3) {Gestaltungslösungen entwickeln,
            die die Nutzungsanforderungen erfüllen};
        \node [block] (5) [textbox] at (-4.5, 0.25) {Gestaltungslösungen aus der
            Benutzerperspektive evaluieren};
        \node [block] (6) [textbox] at (-6, 4.5) {Gestaltungslösung
            erfüllt die Nutzungsanforderungen};
        \draw[arrow] (1.south) to [out=270, in=90] (2.north);
        \draw[arrow, bend left=45] (2.east) to (3.north);
        \draw[arrow, bend left=45] (3.south) to (4.east);
        \draw[arrow, bend left=45] (4.west) to (5.south);
        \draw[arrow] (5.west) to [out=180, in=270] (6.south);
        \draw[arrow, dashed, bend left=45] (5.north) to (2.west);
        \draw[arrow, dashed, bend left=35] (5.north) to (3.north);
        \draw[arrow, dashed, bend left=75] (5.north) to (4.north);
        \node [fill=white, fill opacity=0.7, text width=4cm, text opacity=1] (7) at (-3.5, 2) {Iteration, soweit
            Evaluationsergebnisse Bedarf hierfür aufzeigen};
    }
    \caption{Der menschenzentrierte Gestaltungsprozess \cite{iso-9241-210}}
    \label{fig:din-hcd}
\end{figure}











\chapter{Analyse}

Um entscheidende Rahmenbedingungen für das Projekt festzulegen und
diverse Einflüsse auf die Entwicklung, sowie die spätere Nutzung des Systems
einschätzen zu können, wurde in diesem Kapitel eine ganzheitliche Analyse
durchgeführt. Aufgrund der vielfältigen Veranstaltungstypen wurden die zu
betrachtenden Veranstaltungen eingegrenzt. Zudem wurden Interviews mit
Veranstaltenden (V1-V2) und Teilnehmenden  (T1-T5) durchgeführt, auf dessen
Aussagen sich mit spezifischen IDs im Verlauf der Analyse bezogen wird.
Anschließend wurde eine Organisationsanalyse durchgeführt, um die verschiedenen
Phasen einer Veranstaltung zu eruieren. Die Phasen „Umsetzung“, „Durchführung“
und „Schließung“ wurden hierbei näher analysiert. Basierend auf den Ergebnissen
der Organisationsanalyse wurden eine Benutzer- und Kontextanalyse durchgeführt.
Die Zielgruppen des Frameworks wurden festgelegt und für die Konzeption
relevante Besonderheiten herausgearbeitet. In einer abschließenden
Kontextanalyse wurden die verschiedenen Kontexte der Nutzung des Frameworks
aufgezeigt und auf die mit ihnen einhergehenden Herausforderungen eingegangen.


\section{Datenquellen}

Im Rahmen der Analyse wurden verschiedene Datenquellen genutzt. Es wurde eine
wissenschaftliche Literaturrecherche am 11.08.2021 durchgeführt. Verschiedene
vergleichbare Projekte wurden im Rahmen einer Internetrecherche ermittelt. Unter
anderem wurden die Ausarbeitung und Evaluation der EMI-Award-App als Quellen
genutzt. Zudem wurden Interviews mit Veranstaltenden und Teilnehmenden
durchgeführt. \\
Zur Literaturrecherche wurden die digitalen Bibliotheken ACM Digital
Library \footnote{\url{https://dl.acm.org/}} und Google
Scholar \footnote{\url{https://scholar.google.de/}} genutzt. Es wurden die
folgenden Begriffe aus den Bereichen Event Management (\emph{Event Management,
    Event Evaluation, Event Experience, Event Tracking}) und UX (\emph{Attention
    Economy, Attention Management, Information Overload, Digital Burnout,
    Overchoice, Web Animation, Animation UX, UX Micro Interaction, Animation
    Usability}) zur Suche verwendet. \\
Die Interviews mit Veranstaltenden und Teilnehmenden wurden qualitativ und
semistrukturiert, mithilfe eines dafür entworfenen Interviewleitfadens (siehe
Anhang A), durchgeführt. Bei den Veranstaltenden handelt es sich um Personen mit
praktischer Erfahrung in der Organisation von Veranstaltungen. In Tabelle
\ref{table:ver-soz} und Tabelle \ref{table:teil-soz} sind Details zu den
befragten Personen aufgelistet. Die IDs der interviewten Personen werden im
Verlauf der Analyse referenziert.

\begin{table}[htpb]
    \def\arraystretch{1.25}
    \centering
    \caption{Soziodemografische Daten der Veranstaltenden}
    \label{table:ver-soz}
    \begin{tabular}{lcl}
        \uzlhline
        \uzlemph{ID} & \uzlemph{Alter} &
        \uzlemph{Vorerfahrung}                         \\
        \uzlhline V1 & 40 - 59 J.      & Verschiedene
        Veranstaltungen, verteilt über 11 Jahre        \\
        V2           & 18 - 25 J.      & Mehrere große
        Veranstaltung (>2000 Teilnehmende)             \\
        \uzlhline
    \end{tabular}
\end{table}

\begin{table}[htpb]
    \def\arraystretch{1.25}
    \centering
    \caption{Soziodemografische Daten der Teilnehmenden}
    \label{table:teil-soz}
    \begin{tabular}{lccl}
        \uzlhline
        \uzlemph{ID}                     & \uzlemph{Alter}     &
        \uzlemph{EMI-Award-App genutzt?} & \uzlemph{Tätigkeit}
        \\
        \uzlhline T1                     & 18 - 25 J.          & j
                                         & Student:in              \\
        T2                               & 18 - 25 J.          & j
                                         & Student:in              \\
        T3                               & 18 - 25 J.          & j
                                         & Student:in              \\
        T4                               & 18 - 25 J.          & n
                                         & Student:in              \\
        T5                               & 40 - 59 J.          & n
                                         & Berufstätig             \\
        \uzlhline
    \end{tabular}
\end{table}


\section{Definition: Veranstaltung} \label{sec:analysis-def}

Bevor die verschiedenen Aufgaben, Probleme und Benutzergruppen analysiert werden
können, müssen Umfang und Typ der Veranstaltung festgelegt werden. In der
Literatur sind verschiedene Definitionen zum Begriff „Veranstaltungen“ zu
finden, welche sich in Umfang und Inhalt unterscheiden. Die räumliche und
zeitliche Dimension von Veranstaltungen ist sehr flexibel. \textcite{Getz2007}
definiert Veranstaltungen als zeitliche Phänomene. Bei geplanten Veranstaltungen
wird das Veranstaltungsprogramm oder der Zeitplan im Allgemeinen detailliert
geplant und im Voraus gut bekannt gemacht. Geplante Veranstaltungen sind in der
Regel auch auf bestimmte Orte beschränkt, wobei es sich um eine bestimmte
Einrichtung, eine sehr große Freifläche oder viele Orte handeln kann.
\textcite{Bladen2012} definieren Veranstaltungen als zeitlich begrenzte und
zweckgebundene Zusammenkünfte von Menschen.

Im Rahmen dieser Arbeit werden Veranstaltungen betrachtet, welche in ihrer
zeitlichen Dimension von wenigen Stunden auf mehrere Wochen beschränkt sind und
räumlich verteilt stattfinden. Konkret verteilen sich die Aktivitäten der
Veranstaltung auf einen größeren Raum. Dies kann mindestens die Aufteilung auf
verschiedene Räumlichkeiten sein, bis hin zur Verteilung über verschiedene
Stadtteile.


\section{Organisationsanalyse} \label{sec:analysis-org}

% Masterarbeit_Holtz  Aufgabenanalyse!!!
Die hier analysierte Vorgehensweise in der Organisation von Veranstaltungen
basiert auf dem EMBOK-Model \cite{Silvers2013} und den Aussagen der interviewten
Veranstaltenden. Typischerweise lässt sich die Organisation von Veranstaltungen
in 5 Phasen (s. \autoref{fig:embok-phases}) gliedern: Initiation, Planung,
Umsetzung, Durchführung und Schließung. Die Vorgehensweise in den Phasen
gestaltet sich wie folgt:

\begin{enumerate}
    \setlength{\itemsep}{1em}
    \item \textbf{Initiation} \\
          Es wird geforscht, um ein Konzept zu erstellen und zu validieren.
          Umfang und Kontext sowie Ziele und Aufgaben werden festgesetzt.
    \item \textbf{Planung} \\
          Anforderungen und Spezifikationen werden festgehalten. Hierzu zählen
          die stattfindenden Aktivitäten, sowie die Art der Organisation und
          erforderliche Ressourcen.
    \item \textbf{Umsetzung} \\
          Alle für die Veranstaltung benötigten Waren und Dienstleistungen
          werden in Auftrag gegeben und koordiniert. Der Fokus liegt unter
          anderem auf der Überwachung und Überprüfung des Umfangs und Zeitplans,
          sowie der Kosten und Qualität.
    \item \textbf{Durchführung} \\
          Die veranstaltungsbezogenen Aktivitäten werden aufgenommen. Ab Beginn
          dieser Phase ist der Handlungsrahmen stark eingeschränkt, wodurch der
          Fokus auf der Überwachung der Veranstaltung liegt.
    \item \textbf{Schließung} \\
          Nach Abschluss der Veranstaltung werden Daten gesammelt, ausgewertet
          und weitergegeben. Das Ziel ist die Dokumentation der Erkenntnisse für
          weitere Veranstaltungen.
\end{enumerate}

\begin{figure}[htpb]
    \centering
    \includegraphics[width=\textwidth]{embok_phases.jpg}
    \caption{Die Phasen der Veranstaltungsorganisation \cite{Silvers2013b}. Die
        verschiedenen Phasen (vertikal) bilden die Grundlage einer
        Veranstaltung. Verwoben sind diese mit Wissensdomänen (horizontal).
        Sollte einer der Fäden verschwinden, wird das gesamte Geflecht
        geschwächt.}
    \label{fig:embok-phases}
\end{figure}

% TODO: Abbildungstext in text einbinden
% TODO: Wissensdomänen erklären

Weil das Ziel dieser Arbeit die Unterstützung ab der Umsetzung ist (s.
\autoref{sec:goals}), werden die Phasen „Umsetzung“, „Durchführung“ und
„Schließung“ näher betrachtet.

\subsection{Umsetzung} \label{ssec:analysis-org-umsetzung}

In dieser Phase werden die herausgearbeiteten Pläne umgesetzt. Hierbei werden
Umfang, Zeitpläne, Kosten, Kommunikation und Risiken überwacht und kontrolliert,
um die plangemäße Ausführung sicherzustellen. Weitere Arbeits-/Hilfskräfte
werden engagiert, um bei der Umsetzung mitzuwirken. Für den Erfolg der
Veranstaltung ist die effektive Kommunikation wichtig. Um diese zu
gewährleisten, muss eine Kommunikationsinfrastruktur für Organisierende,
Mitwirkende und externe Dienstleister eingeführt werden. Hilfskräfte müssen in
die Kommunikationsinfrastruktur eingewiesen werden. Aufgrund von mangelnden
technischen Fähigkeiten können bei komplexeren Infrastrukturen Probleme
auftreten (V2).

\subsection{Durchführung} \label{ssec:analysis-org-durchfuehrung}

Mit Beginn der Durchführung ändert sich die Dynamik der Veranstaltung bedeutend.
Aufgrund der Anwesenheit von Teilnehmenden sind tiefgreifende Änderungen nun
nicht mehr möglich und beschränken sich auf die Behebung von kleinen Problemen
(V2). Hingegen wird die wichtigste Aufgabe die Überwachung der Veranstaltung.
Unter besonderer Beobachtung stehen logistische Tätigkeiten, sowie unerwartet
auftretende Probleme~(V1, V2). Diese können u. a. durch Teilnehmende,
stattfindende Aktivitäten oder die Umgebung der Veranstaltung ausgelöst werden.
\\
Während Probleme, welche die Durchführung der Veranstaltung direkt
beeinträchtigen könnten, streng beobachtet und behoben werden, stehen die
Interaktionen und Erfahrungen der Teilnehmenden zunächst im Hintergrund.
Abgesehen von kritischem Feedback, wird die Erfahrung erst während der
Schließung erfasst und ausgewertet (V2).

\subsection{Schließung} \label{ssec:analysis-org-schliessung}

% Bedeutung Evaluation aus Finischen Case S.57-59 Mischung Quantitativ &
% Qualitativ Evaluation S.61

Zur Schließung der Veranstaltung wird die Evaluation zur bedeutendsten Aufgabe.
Feedback und Daten werden von organisierender Seite gesammelt. Hierbei wird
zwischen \textit{weichen} und \textit{harten Daten} unterschieden. Harte Daten
bestehen aus Teilnehmerzahlen, Andrang an verschiedenen Aktivitäten, Dauer,
Einkommen und weiteren zählbaren Merkmalen. Im Vergleich dazu bestehen weiche
Daten u. a. aus Beschwerden, Konflikten, Problemen, Komplimenten, Reaktionen und
Empfehlungen. Mitwirkende, Hilfskräfte und die Organisierenden werden meist dazu
in einer Nachbesprechung befragt (V1, V2). Das Feedback von Teilnehmenden wird
oft nur passiv vor Ort vom Personal erfasst und weitergegeben oder beschränkt
sich auf Familie und Freunde (V2). Die erfassten Erkenntnisse werden
dokumentiert, um auf das Wissen in der nächsten Veranstaltung zurückgreifen zu
können.


\section{Benutzeranalyse} \label{sec:analysis-user}

Um die Funktionen des Frameworks zielgruppengerecht gestalten zu können, werden
in diesem Abschnitt die Benutzergruppen des Frameworks festgelegt und näher
analysiert. Zu den Benutzergruppen gehören Veranstaltende, sowie Teilnehmende
einer Veranstaltung.

\subsection{Veranstaltende} \label{ssec:analysis-user-v}

Inzwischen existieren weltweit tausende Einrichtungen, welche formale
Qualifikationen und Ausbildungen im Veranstaltungsmanagement anbieten. Jedoch
sind diese Qualifikationen nicht einheitlich festgelegt, was zu
unterschiedlichen Schwerpunkten, Umfang, Vermittlungsart und letztendlich
erworbenen Qualifikationen führt \cite{Bladen2012}. In Deutschland gibt es den
anerkannten Ausbildungsberuf des/der Veranstaltungs\-kaufmann/-frau. Die
Aufgaben sind hierbei die Konzipierung und Organisation des kaufmännischen
Aspektes von Veranstaltungen \cite{Kultusministerkonferenz2001}. Die Aufgaben
von Event Manager:innen umfassen zusätzlich die allgemeine Organisation und
Aufgaben im Marketing \cite{BundesagenturfurArbeit2021}. Jedoch ist die
Berufsbezeichnung „Event Manager/in“ rechtlich nicht geschützt. \\
Zudem bedarf es in Deutschland keiner formalen Qualifikation, um eine
Veranstaltung beliebiger Größe zu organisieren. Dementsprechend können keine
bestimmten Qualifikationen für Veranstaltende angenommen werden.
% TODO: Und was bringt das??

\subsection{Teilnehmende} \label{ssec:analysis-user-t}

Die Teilnehmenden einer Veranstaltung unterscheiden sich stark in ihren
soziodemografischen Daten je nach Typ der Veranstaltung. Im Kontext der Arbeit
sind besonders Alter, Technikaffinität und körperliche, seelische, geistige oder
Sinnesbeeinträchtigungen. Für die Betrachtung in dieser Arbeit wird das Alter
der Teilnehmenden auf 16 - 60 J. begrenzt. Zudem wird die sichere grundlegende
Umgangsweise mit einem Smartphone vorausgesetzt.

Da die Literatur zu Event Management ihren Fokus auf die Organisation legt, sind
nur wenig Informationen zur Sicht der Teilnehmenden und ihren Hürden vorhanden.
Als Messwerte für eine erfolgreiche Veranstaltung werden meist
geschäftsrelevante Werte wie z. B. verkaufte Tickets, Teilnehmerzahlen, Umsatz
oder öffentliche Aufmerksamkeit verwendet, welche wenig über die Erfahrung der
Teilnehmenden aussagen. In Interviews wurden die Teilnehmenden gebeten zu
begründen, was eine gewählte Veranstaltung zu ihrem persönlichen Favoriten macht
(s. \autoref{table:teil-fav}).

\begin{table}[htpb]
    \def\arraystretch{1.25}
    \centering
    \caption{Merkmale von positiv empfundenen Veranstaltungen}
    \label{table:teil-fav}
    \begin{tabular}{ll}
        \uzlhline
        \uzlemph{Grund}               & \uzlemph{ID} \\
        \uzlhline  Sozialer Austausch & T1, T2       \\
        Innovative Technik            & T2           \\
        Persönliche Bindung           & T4           \\
        Gute Musik                    & T3           \\
        Einfache Wegfindung           & T5           \\
        Neues Wissen                  & T5           \\
        \uzlhline
    \end{tabular}
\end{table}

\section{Kontextanalyse} \label{sec:analysis-context}

In diesem Abschnitt werden der zeitliche und räumliche Kontext untersucht. Aus
der Beschreibung in \autoref{sec:goals} geht der Fokus auf die Unterstützung von
Veranstaltenden und Teilnehmenden hervor. Die Unterteilung der Benutzergruppen,
wie in \autoref{sec:analysis-user} beschrieben, ist zu beachten. Hieraus ergeben
sich unterschiedliche räumliche und zeitliche Kontexte für Veranstaltende und
Teilnehmende.

Die Unterstützung der Veranstaltenden findet in verschiedenen Phasen der
Organisation statt. Der zeitliche Kontext kann mit Blick auf
\autoref{sec:analysis-org} in „vor, während und nach“ einer Veranstaltung
unterteilt werden. Vor der Durchführung einer Veranstaltung ist insbesondere die
Vorbereitung von Aktivitäten und Absicherung von Dienstleistungen oder Waren
wichtig. Während der Veranstaltung hingegen findet ein starker Fokuswechsel auf
die Beobachtung und schnelle Behebung von Problemen statt. Nach einer
Veranstaltung sind die Datensammlung und Auswertung, sowie der Wissenstransfer
von Bedeutung. \\
Auch beschränken sich die betrachteten Veranstaltungen während der Durchführung
auf den in \autoref{sec:analysis-def} beschriebenen örtlichen Bereich, woraus
sich ein weitestgehend unbeschränkter räumlicher Kontext ergibt. Vor und nach
der Veranstaltung werden komplexe Informationen eingetragen oder ausgewertet. Da
die Bildschirmgröße einen großen Einfluss auf die effiziente Verarbeitung von
Informationen hat \cite{Ni2006}, wird hierfür von einem Arbeitsplatzsystem ausgegangen. Somit
ergibt sich ein räumlich eingeschränkter Kontext vor und nach der Veranstaltung.
Folglich wird während der Veranstaltung ein mobiles System zur Überwachung
benötigt, wobei die Bedingungen vor und nach der Veranstaltung ein stationäres
System befürworten.

Für Teilnehmende ist der Kontext ebenfalls in „vor, während und nach“ der
Veranstaltung aufgeteilt. Der zeitliche Kontext während der Veranstaltung ist
hierbei am bedeutendsten. Auch für Teilnehmende gilt der in
\autoref{sec:analysis-def} beschriebene örtliche Bereich, woraus sich ebenfalls
ein unbeschränkter räumlicher Kontext ergibt. Folglich muss das System für die
Teilnehmenden mobil einsetzbar sein.


\section{Analyse des bestehenden Systems} \label{sec:analysis-old}

An dieser Stelle wird die Implementation und Evaluation der
\textit{EMI-Award-App} vorgestellt, welche die Grundlage des neuen Systems bildet.

% Tech-Stack
%   - Hat sich größten teils bewährt
%   - Web-App mit PWA Funktionalität

% Mockups
%   - Bereits getestet
%   - Grobe Struktur beibehalten

\subsection{Funktionalitäten} \label{ssec:analysis-old-funk}

Die Funktionalitäten der EMI-Award-App gliedern sich in zwei Kategorien:
Funktionalitäten für Teilnehmende und Funktionalitäten für Veranstaltende.
Im Folgenden werden zuerst die Seite der Teilnehmenden beschrieben und
anschließend auf die Möglichkeiten der Veranstaltenden eingegangen.

Die \textit{EMI-Award-App} ermöglicht das augmentierte Besuchen, Einsehen und
Bewerten der EMI-Award-Projekte und bildet die Grundlage dieser Arbeit (s.
\autoref{sec:goals}). Speziell werden dafür an den Projekt-/Stationsstandorten
QR-Codes bereitgestellt, welche mithilfe eines QR-Code-Readers oder der
Kamerafunktion der App gescannt werden können (s. \autoref{fig:emi-qr-code}).

% Speziell wurden dafür an verschiedenen Standorten Schilder
% mit QR-Codes angebracht, welche mit einem QR-Code-Reader oder der App gescannt
% werden konnten (s. \autoref{fig:emi-qr-code}).

\begin{figure}[htpb]
    \centering
    \includegraphics{emi_schild.png}
    \caption{Schild eines Projektstandortes mit QR-Code}
    \label{fig:emi-qr-code}
\end{figure}

Beim ersten Aufruf der App wird dem Nutzenden eine kleine Einführung in die App
und ihren Hintergrund gegeben. Dies geschieht über eine interaktive Slideshow.
Nach dem Bestätigen der letzten Slide werden Nutzende zur interaktiven Karte
geführt (s. \autoref{fig:emi-intro-map}). Auf dieser werden die Projektstandorte
durch verschiedenfarbige Icons von Planeten (\textgraphics{emi_planet.png})
dargestellt. Durch das Antippen eines Standorts wird eine Kurzinformation, in
Form eines Pop-ups, zum jeweiligen Projekt angezeigt. Außerdem wird der
Standort des Nutzenden angezeigt, insofern die Berechtigungen dazu gegeben
werden.

\begin{figure}[htpb]
    \begin{minipage}{.5\textwidth}
        \centering
        \includegraphics[width=.6\linewidth]{emi_prolog.png}
    \end{minipage}%
    \begin{minipage}{.5\textwidth}
        \centering
        \includegraphics[width=.6\linewidth]{emi_map.png}
    \end{minipage}
    \caption{Einführung und interaktive Karte mit Projektstandorten}
    \label{fig:emi-intro-map}
\end{figure}

Um die vollständigen Informationen eines Projekts einzusehen, muss der
Standort erst virtuell besucht werden. Dies geschieht über das Scannen des
QR-Codes auf dem Schild des Standorts (s. \autoref{fig:emi-qr-code}) oder der
manuellen Eingabe eines spezifischen Codes. Nach erfolgreichem Scannen oder
Eingeben des Codes wird eine virtuelle Szene angezeigt. In dieser muss ein
Planet gesucht und angetippt werden, um den Vorgang abzuschließen (s.
\autoref{fig:emi-ar}). Je nach benutztem Gerät wird die Szene per Augmented
Reality (AR) oder Virtual Reality angezeigt (VR). Zudem wird die Anzahl der
besuchten Standorte in der Karten- und Projektansicht anhand eines
Fortschrittsbalkens angezeigt. Die vollständigen Informationen der Projekte
enthalten mediale Inhalte, eine detaillierte Beschreibung und die Autoren des
Projekts. Zusätzlich kann das Projekt mit einer vorgegebenen Auswahl bewertet
werden (s. \autoref{fig:emi-project}).

\begin{figure}[htpb]
    \begin{minipage}{.5\textwidth}
        \centering
        \includegraphics[width=.6\linewidth]{emi_ar-1.png}
    \end{minipage}%
    \begin{minipage}{.5\textwidth}
        \centering
        \includegraphics[width=.6\linewidth]{emi_ar-2.png}
    \end{minipage}
    \caption{AR/VR Szene während des virtuellen Besuchs}
    \label{fig:emi-ar}
\end{figure}

\begin{figure}[htpb]
    \begin{minipage}{.5\textwidth}
        \centering
        \includegraphics[width=.6\linewidth]{emi_project-1.png}
    \end{minipage}%
    \begin{minipage}{.5\textwidth}
        \centering
        \includegraphics[width=.6\linewidth]{emi_project-2.png}
    \end{minipage}
    \caption{Einzelansicht eines Projekts}
    \label{fig:emi-project}
\end{figure}

Des Weiteren können verschiedene Abzeichen für bestimmte Aktionen erhalten
werden. Aktionen sind z. B. das Besuchen einer bestimmten Anzahl von Projekten
oder Benutzen von bestimmten Funktionen. Alle Abzeichen und ihr Fortschritt
können in einer Übersicht eingesehen werden (s. \autoref{fig:emi-achievments}).
Bestimmte Abzeichen werden in mehreren Stufen freigeschaltet oder bleiben bis
zum Erhalt verborgen. \\
Zudem werden die gesammelten „Awardteile“ angezeigt. Awardteile sind sammelbare
Objekte, welche auf der Karte mit einem Meteorit-Symbol
(\textgraphics{emi_meteorit.png}) gekennzeichnet sind. Beim Annähern an die
gezeigten Standorte wird eine Schaltfläche hervorgehoben, welche durch Antippen
das entsprechende Awardteil sammelt.

\begin{figure}[htpb]
    \centering
    \includegraphics[width=.3\textwidth]{emi_achievements.png}
    \caption{Abzeichen- und Awardteilansicht}
    \label{fig:emi-achievments}
\end{figure}

Außerdem werden die genannten Funktionen in der App durch eine Slideshow
erklärt, welche jederzeit über die Navigationsleiste erreichbar ist (s.
\autoref{fig:emi-help}).

\begin{figure}[htpb]
    \centering
    \includegraphics[width=.3\textwidth]{emi_help.png}
    \caption{Eine Seite der In-App-Anleitung}
    \label{fig:emi-help}
\end{figure}

Aus Veranstaltendensicht kann die EMI-Award-App rudimentär verwaltet
werden. Stationen, Abzeichen, Hilfseinträge, die Einführung und sammelbare
Objekte lassen sich über eine Web-Oberfläche eintragen und jederzeit verändern.
Jedoch werden Besuche und gesammelte Abzeichen vom System nicht erfasst und
sind somit nicht auswertbar. Auch die gesammelten Daten, darunter z. B. die
Anzahl der registrierten Teilnehmenden, sind nicht einsehbar.

Zusammenfassend ergeben sich daraus die in \autoref{table:emi-func} dargelegten
Funktionalitäten.

\begin{table}[htpb]
    \def\arraystretch{1.25}
    \centering
    \caption{Funktionalitäten der EMI-Award-App}
    \label{table:emi-func}
    \begin{tabular}{ll}
        \uzlhline
        \uzlemph{ID} & \uzlemph{Funktionalität}                 \\
        \uzlhline%
        FT-Emi-1     & Interaktive Karte mit Stationen          \\
        FT-Emi-2     & Auflistung der Stationen                 \\
        FT-Emi-3     & Virtuelles Besuchen per QR-Code          \\
        FT-Emi-4     & Abzeichen                                \\
        FT-Emi-5     & Bedienungshilfe                          \\
        FT-Emi-6     & Einleitende Slideshow                    \\
        FT-Emi-7     & Rudimentäre Verwaltung der Informationen \\
        \uzlhline
    \end{tabular}
\end{table}

\subsection{Benutzeroberfläche} \label{ssec:analysis-old-ui}

Im Allgemeinen ist die verwendete Schriftgröße der Benutzeroberfläche zu klein.
Die empfohlene Mindestschriftgröße beträgt 16 px für Smartphones im Abstand von
30 cm \cite{DIN1450}. Einige Teile der Benutzeroberfläche verwenden jedoch eine
Schriftgröße von nur 12 px.

Das Karten-Pop-up (s. \autoref{fig:emi-intro-map}) ist von diesem Problem
betroffen. Es verwendet eine Schriftgröße von 12 px und ist somit deutlich zu
klein. Des Weiteren nutzen Titel und Kurzbeschreibung identische Typografie. Der
Abstand ist ebenfalls identisch mit dem Zeilenabstand der Texte. Es liegt somit
kein visueller Unterschied zwischen Titel und Kurzbeschreibung vor. \\
Außerdem gibt es keinen Indikator dafür, ob die Station bereits besucht wurde.
Somit müssen Nutzer:innen sich den Besucht-Status in der Stationsliste anschauen
und merken, um auf der Karte darüber informiert zu sein. Dies steht im Gegensatz
zur 6. Usability Heuristik nach \textcite{Nielsen1994}, welche besagt, dass
Nutzer:innen nicht gezwungen sein sollten, sich Informationen zwischen Ansichten
zu merken. Stattdessen sollten die Informationen direkt ersichtlich sein.


\subsection{Technische Umsetzung} \label{ssec:analysis-old-tech}

Die EMI-Award-App wurde als progressive Web-App (PWA) mit dem \textit{Vue.js}
Framework realisiert und war unter \url{https://app.emi-award.de} verfügbar. Die
Web-Oberfläche für Veranstaltende wurde basierend auf \textit{Strapi CMS}
entwickelt. Da der Fokus des zugehörigen Bachelorprojekts auf der Entwicklung
einer App für den EMI-Award lag, wurden die Funktionalitäten speziell und
nicht-anpassbar auf diesen zugeschnitten. Somit müssen sämtliche Teilsysteme
überarbeitet werden, um diese für andere Veranstaltungen zu verallgemeinern.


\section{Problemanalyse} \label{sec:analysis-problems}

Während Planung, Durchführung und Nachbereitung von Veranstaltungen können
Probleme auftreten, welche die Organisation erschweren. Ähnlich treten bei der
Teilnahme Probleme auf, welche die Erfahrung beeinträchtigen können.

\subsection{Organisation} \label{ssec:analysis-problems-orga}

In der Organisation von Veranstaltungen treten vor allem Probleme in der
Kommunikation auf. Je nach Größe der Veranstaltung können eine Vielzahl an
Menschen an der Organisation beteiligt sein. Meist sind verschiedene digitale
Tools oder Plattformen zur Kommunikation erforderlich, um sich mit allen
Veranstaltenden zu verständigen (V2). Hier kann es zu Herausforderungen kommen,
auch Menschen mit geringem Technikverständnis einzuweisen (V2). Aber auch der
Kontakt zu Teilnehmenden spielt eine große Rolle. Jedoch ist der Kontakt während
einer Veranstaltung nur gering (V1, V2). Nach einer Veranstaltung fällt der
Kontakt ebenfalls sehr schwer (V1) oder wird ausgelassen (V2).

Weitere Probleme treten in der Beeinflussung von Veranstaltungen während ihrer
Durchführung auf. Es besteht der Bedarf kleine Änderungen noch einfacher
vornehmen zu können (V1). Zudem mangelt es bei räumlich verteilten
Veranstaltungen an Möglichkeiten ein Gruppengefühl zu erzeugen (V1).

Die Übersicht über die Veranstaltung stellt ebenfalls ein Problem dar. Gerade
bei räumlich größeren Veranstaltungen gibt es keine direkte Übersicht über die
Teilnehmenden (V1). Oft werden für die Datensammlung nur Schätzungen verwendet
(V2). Jedoch ist die Reichweite ein wichtiger Indikator für Veranstaltende (V1).
Allgemein besteht die Datensammlung in der Nachbesprechung aus Aussagen von
Veranstaltenden oder deren Angehörigen und Freunden (V2).

\subsection{Teilnahme} \label{ssec:analysis-problems-teil}

Durch die Interviews mit Teilnehmenden konnte festgestellt werden, dass
Schwierigkeiten in der Navigation von Veranstaltungen zu finden sind. Daher
besteht die Nachfrage nach einer Unterstützung in der Wegfindung und eine
Übersicht über die Veranstaltung (T1, T2, T5). Diese Unterstützung kann zum
Beispiel ein Event-Assistent umfassen (T2). Des Weiteren konnte sowohl
durch Teilnehmende, als auch durch Veranstaltende erfasst werden, dass Personen
mit körperlichen Einschränkungen durch technische Unterstützung besser oder
womöglich überhaupt an Events teilnehmen können (bspw. Gebärdensprache in Form
von Videos, Videos für Blinde, etc.).


\section{Formalisierte Anforderungen} \label{sec:analysis-anf}

In diesem Abschnitt werden die systematisch formalisierten Anforderungen
dargelegt, welche die Ergebnisse der gesamten Analyse nach
\textcite{Balzert2009} zusammenfassen. Zu Beginn werden Visionen und darauf
aufbauende Ziele definiert. Zudem werden Rahmenbedingungen und Kontext des
Systems aufgezeigt. Anschließend werden die daraus resultierenden funktionalen
und qualitativen Anforderungen festgelegt.

\subsection{Vision und Ziele}

Die Grundlage der Anforderungen bilden die Visionen und Ziele des Systems. Das
Festlegen dieser erlaubt eine Überprüfung der Zielgerichtetheit der
Anforderungen \cite{Balzert2009}. Diese leiten sich aus den Zielen der Arbeit
(\autoref{sec:goals}), sowie der vorangegangenen Analyse der Organisation,
Benutzenden und Problemen ab. Den Anfang bilden die Visionen, welche
realitätsnah Vorstellungen der Zukunft darstellen.

\setanf{V}
\begin{center}
    \def\arraystretch{1.5}
    \begin{tabular}{m{0.08\textwidth}m{0.85\textwidth}}
        \uzlhline
        \anfrow & Veranstaltende sind besser in der Lage, ihre
        Veranstaltung zu planen, überblicken und auszuwerten.
        \\
        \anfrow & Veranstaltende sind besser in der Lage, mit
        Teilnehmenden zu kommunizieren.
        \\
        \anfrow & Teilnehmende werden motiviert und
        unterstützt, sich in der Veranstaltung zurechtzufinden.
        \\
        \anfrow & Teilnehmende können auf Wunsch der Veranstaltenden dazu
        motiviert werden, sich sozial auszutauschen
        \\
        \uzlhline
    \end{tabular}
\end{center}

Auf Basis der Visionen werden Ziele formuliert, welche die Visionen
operationalisieren. Diese folgen dabei den standardisierten Regeln zur
Formulierung von Zielen \cite{Pohl2008}.

\setanf{Z}
\begin{center}
    \def\arraystretch{1.5}
    \begin{longtable}{m{0.08\textwidth}m{0.85\textwidth}}
        \uzlhline
        \anfrow    & Veranstaltende sollen jederzeit in der Lage sein, Informationen
        zentral einzupflegen und diese zu verändern, um effektiv auf
        Veränderungen einzugehen.
        \\
        \anfrow    & Veranstaltende und Teilnehmende sollen ab Beginn der Veranstaltung in der
        Lage sein, besser miteinander zu kommunizieren.                                        \\
        \anfsubrow & Veranstaltende sollen ab Beginn der Veranstaltung in der
        Lage sein, Teilnehmenden Benachrichtigungen zu senden, um diesen zeitnah
        mit wichtigen Informationen versorgen zu können.                                       \\
        \anfsubrow & Veranstaltende sollen ab Beginn der Veranstaltung in der
        Lage sein, Teilnehmende jederzeit um Feedback zu fragen, um Probleme
        frühzeitig zu erkennen.
        \\
        \anfsubrow & Teilnehmende sollen ab Beginn der Veranstaltung in der
        Lage sein, Veranstaltende unkompliziert und jederzeit zu kontaktieren.
        \\
        \anfrow    & Teilnehmende sollen ab Beginn der Veranstaltung in der
        Lage sein, sich über die Aktivitäten der Veranstaltung zu informieren,
        um die Navigation der Veranstaltung zu erleichtern.                                    \\
        \anfrow    & Teilnehmende sollen ab Beginn der Veranstaltung in der
        Lage sein, bei Schwierigkeiten schnell an hilfreiche Informationen zu
        gelangen.                                                                              \\
        \anfrow    & Teilnehmende sollen ab Beginn der Veranstaltung in der Lage
        sein, für ihre aktive Teilnahme belohnt zu werden.
        \\
        \anfrow    & Teilnehmende sollen in der Lage
        sein, die veranstaltungsbezogenen Aktivitäten mit anderen Personen gemeinsam abzuschließen
        \\
        \anfrow    & Das System soll Informationen zugänglich präsentieren.
        \\
        \anfrow    & Veranstaltende sollen in der Lage sein, das Design des
        Systems auf ihre Veranstaltung anzupassen.
        \\
        \anfrow    & Veranstaltende sollen jederzeit in der Lage sein, vom
        System gesammelte Daten übersichtlich und strukturiert einzusehen.                     \\
        \uzlhline
    \end{longtable}
\end{center}
\vspace{-3cm}

\subsection{Rahmenbedingungen}

Die Rahmenbedingungen legen organisatorische und technische Restriktionen für
das System oder den Entwicklungsprozess fest \cite{Balzert2009}. Die Bedingungen
wurden aus der Benutzer und Kontextanalyse abgeleitet.

\setanf{R}
\begin{center}
    \def\arraystretch{1.5}
    \begin{longtable}{m{0.08\textwidth}m{0.85\textwidth}}
        \uzlhline
        \anfrow & Das System ist eine Web-Anwendung.
        \\
        \anfrow & Die Zielgruppen sind Personen, welche Veranstaltungen
        organisieren und Teilnehmende dieser Veranstaltungen. Die
        Nutzungsgruppen wurden in \autoref{sec:analysis-user} definiert.
        \\
        \anfrow & Das System wird von Veranstaltenden in einem
        Arbeitsplatzsystemkontext genutzt und von Teilnehmenden vorwiegend in
        einem mobilen Kontext genutzt.                                  \\
        \anfrow & Das System wird über die Länge der Veranstaltung im
        Dauerbetrieb laufen.                                            \\
        \anfrow & Das System soll auf Wunsch auch nach der
        Veranstaltung noch für beliebige Zeit erreichbar sein.          \\
        \anfrow & Das System muss unbeaufsichtigt zuverlässig lauffähig
        sein.                                                           \\
        \anfrow & Auf den Zielmaschinen verwendete Software:
        \newline
        Client:
        Moderne marktführende Webbrowser (Chrome, Firefox, Edge, Safari)
        \\
        \uzlhline
    \end{longtable}
\end{center}
\vspace{-3cm}

\subsection{Kontext und Überblick}

Jedes System ist in einer technischen Umgebung eingebettet \cite{Balzert2009}.
Im Folgenden wird Bezug auf das aktuelle System genommen.

\setanf{K}
\begin{center}
    \def\arraystretch{1.5}
    \begin{tabular}{m{0.08\textwidth}m{0.85\textwidth}}
        \uzlhline
        \anfrow & Das aktuelle System ist eine Web-Anwendung, welche die
        Grundlage des neuen Systems darstellt.
        \\
        \uzlhline
    \end{tabular}
\end{center}
\vspace{-1cm}

\subsection{Funktionale Anforderungen}

Die funktionalen Anforderungen halten die Kernfunktionalitäten des Systems fest
\cite{Balzert2009}. Diese ergeben sich aus allen Teilanalysen und den
festgelegten Zielen. Um eine eindeutige Semantik bei natürlichsprachlichen
Anforderungen zu gewährleisten, wird eine Anforderungsschablone (s. \autoref{fig:anf-schablone}) verwendet,
welche die Anforderungen vereinheitlicht \cite{Balzert2009}.

\begin{figure}[htpb]
    \renewcommand\baselinestretch{1}
    \centering
    \tikzset{
        textbox/.style={
                minimum height=1.75cm,
                text centered,
                align=center
            },
        small/.style={
                text width=1.9cm,
            },
        medium/.style={
                text width=2.5cm,
            },
        big/.style={
                text width=2.92cm,
            },
        arrow/.style={
                -
            },
        every node/.style={scale=0.8},
    }
    \tikz [thesis box shapes, baseline, anchor=base, scale=0.8]{
        \coordinate (origin);

        \node [block] (1a) [textbox, medium, right=of origin] {\textbf{Element}\linebreak„Die Komponente <Name>“};
        \node [block] (1b) [textbox, medium, above=0.35cm of 1a] {\textbf{Element}\linebreak„Das System“};

        \coordinate [right=0.175cm of {$(1a.east)!0.5!(1b.east)$}] (1X);

        \node [block] (2a) [textbox, small, right=0.175cm of 1X] {\textbf{Mittlere Priorität}\linebreak„soll“};
        \node [block] (2b) [textbox, small, above=0.35cm of 2a] {\textbf{Hohe Priorität}\linebreak„muss“};
        \node [block] (2c) [textbox, small, below=0.35cm of 2a] {\textbf{Niedrige Priorität}\linebreak„sollte in Zukunft“};

        \coordinate [right=0.175cm of 2a] (2X);

        \node [block] (3a) [textbox, big, right=0.175cm of 2X] {\textbf{Benutzerinteraktion}\linebreak„<Wem> die Möglichkeit bieten“};
        \node [block] (3b) [textbox, big, above=0.35cm of 3a] {\textbf{Selbstständige Systemaktivität}\linebreak„“};
        \node [block] (3c) [textbox, big, below=0.35cm of 3a] {\textbf{Schnittstellenanforderungen}\linebreak„fähig sein“};

        \node [block] (4) [textbox, medium, right=0.35cm of 3a] {\textbf{Bezug}\linebreak„<Objekt \& Ergänzung des Objektes>“};

        \node [block] (5) [textbox, medium, right=0.35cm of 4, yshift=-2cm] {\textbf{Bei Qualitätsanforderungen}\linebreak„<Qualität>“};

        \node [block] (6) [textbox, medium, right=0.35cm of 5, yshift=2cm] {\textbf{Funktionalität}\linebreak„<Prozesswort>“};

        \draw[arrow] (1a.east) to (1X);
        \draw[arrow] (1b.east) to (1X);
        \draw[arrow] (1X) to (2a.west);
        \draw[arrow] (1X) to (2b.west);
        \draw[arrow] (1X) to (2c.west);

        \draw[arrow] (2a.east) to (2X);
        \draw[arrow] (2b.east) to (2X);
        \draw[arrow] (2c.east) to (2X);
        \draw[arrow] (2X) to (3a.west);
        \draw[arrow] (2X) to (3b.west);
        \draw[arrow] (2X) to (3c.west);

        \draw[arrow] (3a.east) to (4.west);
        \draw[arrow] (3b.east) to (4.west);
        \draw[arrow] (3c.east) to (4.west);

        \draw[arrow] (4.east) to (5.west);
        \draw[arrow] (4.east) to (6.west);

        \draw[arrow] (5.east) to (6.west);
    }
    \caption{Anforderungsschablone \cite{Balzert2009}}
    \label{fig:anf-schablone}
\end{figure}

% TODO: Anfang und Ende Funktion als Anforderung

\setanf{F}
\begin{center}
    \def\arraystretch{1.5}
    \begin{longtable}{m{0.08\textwidth}m{0.85\textwidth}}
        \uzlhline
        \anfrow    & Das System \textit{muss} Veranstaltenden die
        Möglichkeit bieten, veranstaltungsrelevante Informationen einzutragen
        und jederzeit zu verändern (\anfref{Z10}).                                   \\
        \anfsubrow & Das System \textit{muss} Veranstaltenden die
        Möglichkeit bieten, Stationen einzutragen und jederzeit zu verändern.        \\
        \anfsubrow & Das System \textit{muss} Veranstaltenden die
        Möglichkeit bieten, Abzeichen einzutragen und jederzeit zu verändern.        \\
        \anfsubrow & Das System \textit{muss} Veranstaltenden die
        Möglichkeit bieten, Hilfe (FAQ) jederzeit einzutragen oder zu verändern.
        \\
        \anfsubrow & Das System \textit{muss} Veranstaltenden die
        Möglichkeit bieten, einführende Folien (Intro) jederzeit einzutragen
        oder zu verändern.
        \\
        \anfrow    & Das System \textit{muss} Veranstaltenden die Möglichkeit
        bieten, Statistiken zu den veranstaltungsbezogenen Aktivitäten
        einzusehen (\anfref{Z80}).                                                   \\
        \anfrow    & Das System \textit{muss} Teilnehmenden ab Beginn der
        Veranstaltung die Möglichkeit bieten, jederzeit Informationen
        einzusehen (\anfref{Z30}).
        \\
        \anfrow    & Das System \textit{muss} Teilnehmenden ab Beginn der
        Veranstaltung die Möglichkeit bieten, eine Einführung in den Kontext der
        Veranstaltung zu erhalten (\anfref{Z30}).
        \\
        \anfrow    & Das System \textit{muss} Teilnehmenden ab Beginn der
        Veranstaltung die Möglichkeit bieten, jederzeit hilfreiche Einträge bei Fragen einzusehen (\anfref{Z40}).
        \\
        \anfrow    & Das System \textit{muss} Teilnehmenden ab Beginn der
        Veranstaltung die Möglichkeit bieten, Abzeichen für bestimmte Handlungen
        zu verleihen (\anfref{Z50}).                                                 \\
        \anfrow    & Das System \textit{muss} Veranstaltenden die
        Möglichkeit bieten, Teilnehmenden jederzeit Benachrichtigungen zu
        senden (s. \ssecref{ssec:analysis-problems-orga}, \anfref{Z21}).
        \\
        \anfrow    & Das System \textit{soll} Veranstaltenden die Möglichkeit
        bieten, Teilnehmende jederzeit um Feedback bitten (s.
        \ssecref{ssec:analysis-problems-orga}, \anfref{Z22}).                        \\
        \anfrow    & Das System \textit{soll} Veranstaltenden die
        Möglichkeit bieten, die Rahmendaten der Veranstaltung jederzeit
        einzutragen oder zu ändern (\anfref{Z10}).
        \\
        \anfrow    & Das System \textit{soll} Teilnehmenden die Möglichkeit
        bieten, als Gruppe Stationen zu besuchen und Abzeichen abzuschließen.        \\
        \anfrow    & Das System \textit{sollte in Zukunft} fähig sein,
        Informationen in verschiedenen Sprachen darzustellen (\anfref{Z70}).         \\
        \anfrow    & Das System \textit{sollte in Zukunft} Teilnehmenden die
        Möglichkeit bieten, Veranstaltende jederzeit zu kontaktieren (\anfref{Z23}). \\
        \anfrow    & Das System \textit{sollte in Zukunft} umfassende
        Möglichkeiten zur Anpassung der Signalfarbe an die eigene Veranstaltung
        zulassen (\anfref{Z70}).
        \\
        \uzlhline
    \end{longtable}
\end{center}
\vspace{-3cm}

\subsection{Qualitätsanforderungen}

Abschließend werden Qualitätsanforderungen (auch nicht-funktionale
Anforderungen) festhalten. Diese legen qualitative oder quantitative
Eigenschaften des Systems fest \cite{Balzert2009}. Auch hier wird sich stark an
der Anforderungsschablone (s. \autoref{fig:anf-schablone}) orientiert.

\setanf{Q}
\begin{center}
    \def\arraystretch{1.5}
    \begin{longtable}{m{0.08\textwidth}m{0.85\textwidth}}
        \uzlhline
        \anfrow & Das System \textit{muss} den Grundsätzen der DIN EN ISO
        9241-110:2019-09 (Ergonomie der Mensch-System-Interaktion - Teil 110:
        Interaktionsprinzipien) folgen (\cite{iso-9241-210}).                                                              \\
        \anfrow & Das System \textit{muss} die definierten Nutzungsklassen
        aus \autoref{sec:analysis-user} (Veranstaltende und Teilnehmende)
        unterscheiden und die dazugehörigen Zugriffsrechte sicherstellen.
        \\
        \anfrow & Das System \textit{muss} zuverlässig und ohne Störung im Dauerbetrieb laufen.
        \\
        \anfrow & Das System \textit{muss} die zehn Usability Heuristiken
        \cite{Nielsen1994} beachten. (\anfref{Z60})
        \\
        \anfrow & Das System \textit{soll} die Richtlinien des Deutschen
        Blinden- und Sehbehindertenverbandes berücksichtigen \cite{DBSV2022}.                                              \\
        \anfrow & Das System \textit{soll} beim Zugriff über das Internet eine gesicherte Übertragung (HTTPS) ermöglichen.
        \\
        \anfrow & Das System \textit{soll} alle Benutzerinteraktionen in unter fünf Sekunden ausführen.
        \\
        \anfrow & Das System \textit{sollte in Zukunft} weitestgehend offline
        genutzt werden können.
        \\
        \uzlhline
    \end{longtable}
\end{center}

% Während der Veranstaltung Fokus Überwachung und Sicherstellung der Aktivitäten
% -> Die Stationsdashboards mit Nachrichten, Informationen und Übersicht zur
% Überwachung

% Mobiler Einsatz -> Betriebssysteme für Handys
%   - iOS vs Android https://gs.statcounter.com/os-market-share/mobile/europe/
%   - PWA / Web-App

% Stabilität und Zuverlässigkeit *sehr* wichtig
%   - Ausfall kann gesamtes Event lahmlegen
%   - Bei hoher Teilnehmeranzahl hohe Last

% Warum inclusive Design
%   - Menschen aller gesellschaftlichen Schichten nehmen an Veranstaltungen teil
%   - Jedem sollte die Chance geboten werden teilzunehmen

% Gruppengefühl, Soziale Interaktion -> Gruppenfunktion, Reaktionen

% Bessere Kommunikation -> Stationsdashboard

\chapter{Konzeption}

Nisi et sed provident esse accusamus consequuntur praesentium qui. Eaque vel non dolores aliquam fuga voluptas quia sit. Vel ut rem et in quis quo inventore quidem. Enim quam voluptatum atque et. Consequuntur repellendus quia voluptate vel quia et suscipit soluta. Fugiat iste corporis voluptatem molestiae.

\section{Systemarchitektur}

% Speichern von Nutzerdaten ohne konkrete Anmeldung

% Usability von interaktiven Karten
% - Alter
% - Behinderung
% - Technikaffinität

% Usability QR-Code Reader

\section{Frontend}

% inclusive Design
%   - Mehrsprachig
%   - A11y (Aria, W3C Empfehlungen)

% Auf "Refactoring UI"-Standards geachtet

% Tailwindcss Design-System verwendet

\section{Backend}

% Komplexe Modellierung: Gruppen / Einzel

% Darstellung der Datenbank Relation

% Strukturierung der API
%   - Gruppen API
%   - Besuchs API
%   - Benachrichtigungs API
%   - Abzeichen API

\section{Implikation für Implementierung}

% Gedanken zur technischen Umsetzung der EMI-App
%   - Vor-/Nachteile Web/Native
%   - Library Entscheidungen

\chapter{Implementierung}

In diesem Kapitel wird die Implementierung der im letzten Kapitel erarbeiteten
Konzeption für Frontend und Backend der Anwendung beschrieben.

% Um die Softwarequalität bei der gegebenen Komplexität des Frameworks
% garantieren zu können, wurden Softwaretests eingesetzt, welche die
% grundlegende Logik der verschiedenen Funktionen des Backends überprüfen
% sollten. Des Weiteren wurde ein hohes Maß an Automation angestrebt, um die
% Softwarequalität kontinuierlich zu überprüfen und die verwendete Zeit für
% repetitive Aufgaben zu minimieren.

% Starker Fokus auf Automation
%   - Github Actions
%   - Docker
%       - Compose
%       - Automatisierte Builds
%       - Watchtower

\section{Frontend}

% Vue 3
%   - vue-cli
%   - SFC
%   - MVC
%   - Style Guide / Best Practices
%       - script setup (recommended
%         https://v3.vuejs.org/api/sfc-script-setup.html)
%   - Composition API

% Typescript

% Genutzte Libraries:
%   - Axios
%   - MapboxGlJS
%   - marked (Markdown) JSDOMPurify
%   - Iconify
%   - Swiper
%   - Socket.IO

% Gliederung:
%   - Controller
%   - Service
%   - Repository

% Kamera Eigen-Implementierung

% Mapbox Eigen-Implementierung

% Modal Eigen-Implementierung

% Service Worker - Push-Benachrichtigungen

% Optimierungen:
%   - Route Splitting
%   - PWA bzw. Service Worker

\section{Backend}

% Strapi
%   - Mangelhafte Dokumentation
%   - Mangelhaftes Tooling

% Group Plugin Implementation

% Location Picker Feld Plugin Implementation

% Verschiedene APIs
%   - completion
%   - visit
%   - notification
%   - feedback

% Authentifizierung

% Socket.IO

% Dashboard

% API-Tests mit Jest

\chapter{Dialogbeispiel} \label{chapter:dialog}

\chapter{Evaluation}
\section{Teilnehmende}
\subsection{Methodik}
\subsection{Ergebnisse}
\section{Veranstaltende}
\subsection{Methodik}
\subsection{Ergebnisse}

\chapter{Zusammenfassung}

\section{Diskussion}

% Strapi Beschwerden
%   - Entwicker Erfahrung
%       - Schlechte Plugin Dokumentation
%       - Schlechte Helper-Plugin Dokumentation
%       - Tooling Probleme
%   - Rechtesystem
%       - Rechte der Gruppen nicht einstellbar
%       - Nur durch Aufpreis

\appendix

\setlength{\parskip}{2pt}

\chapter{Interviewleitpfäden} \label{appendix:interview}

\section{Teilnehmende}


\textbf{\large Einstieg}

\begin{enumerate}[noitemsep,topsep=0pt]
    \item Begrüßung und Danken für die Zeit
    \item Kurzer Umriss des Themas
    \item Ist die EMI-Award-App bekannt?
    \item Alter, Studiengang / Tätigkeit
    \item Kontaktmöglichkeit im Nachhinein
    \item Datenschutz
\end{enumerate}


\textbf{\large Einstiegsfrage}

\begin{enumerate}[noitemsep,topsep=0pt]
    \item {
        Welche von Ihnen besuchte Veranstaltung ist Ihr persönlicher Favorit?
        \begin{enumerate}[noitemsep,topsep=0pt]
            \item Berücksichtigen: Covid, lange keine Präsenz Veranstaltungen
            \item Details zur Veranstaltung erfragen, falls nicht bekannt
        \end{enumerate}
        }
    \item Was macht diese Veranstaltung zu ihrem persönlichen Favoriten?
\end{enumerate}


\textbf{\large Schlüsselfragen}

\textbf{Frage 1:} Wenn Sie sich etwas hilfreiches begleitend zur Veranstaltung wünschen könnten, was würde dies sein?

\textbf{Frage 2:} Welchen Technologien sind Sie auf Veranstaltungen schon begegnet? (Oder auch nicht)
\begin{itemize}[noitemsep,topsep=0pt]
    \item Hat Ihnen die Technologie auf der Veranstaltung geholfen?
    \item Wenn ja, wie? Sonst, warum nicht?
\end{itemize}

\textbf{Frage 3:} Wie wichtig ist Ihnen der soziale Austausch mit anderen Teilnehmenden während einer Veranstaltung?
\begin{itemize}[noitemsep,topsep=0pt]
    \item Wie verändert sich dies mit dem Typ der Veranstaltung?
\end{itemize}

\textbf{Frage 4:} Welchen (multi-/medialen) Weg bevorzugen Sie um Informationen aufzunehmen? (Videos schauen, Berichte/Artikel lesen, Podcasts hören, ...)
\begin{itemize}[noitemsep,topsep=0pt]
    \item Wie verändert sich dies in einem lehrreichen / unterhaltenden Setting?
\end{itemize}


\textbf{\large Abschluss}

\begin{enumerate}[noitemsep,topsep=0pt]
    \item Nochmals für die Zeit Danken
    \item Kontaktmöglichkeit im Nachhinein
    \item Verabschiedung
\end{enumerate}


\section{Veranstaltende}

\textbf{\large Einstieg}

\begin{enumerate}[noitemsep,topsep=0pt]
    \item Begrüßung und Danken für die Zeit
    \item Kurzer Umriss des Themas
    \item Vorerfahrung (Wie viele Veranstaltungen? Wie Groß?)
    \item Kontaktmöglichkeit im Nachhinein
    \item Datenschutz
\end{enumerate}

\textbf{\large Einstiegsfrage: Phase Organisation}

\begin{enumerate}[noitemsep,topsep=0pt]
    \item  Welche von Ihnen (mit)organisierte Veranstaltung ist Ihr persönlicher
          Favorit?
    \item {
          Beschreiben Sie grob den Ablauf bei der Planung der Veranstaltung
          \begin{enumerate}[noitemsep,topsep=0pt]
              \item Ob und wie wird Feedback von Teilnehmenden während Veranstaltung gesammelt?
              \item Ob und wie Kontakt zu Teilnehmenden während Veranstaltung?
              \item Ob und wie Kontakt zu Teilnehmenden nach Veranstaltung?
              \item Welche Priorität hat die Zugänglichkeit der Veranstaltung?
              \item Was wird für die Zugänglichkeit getan?
          \end{enumerate}
          }
\end{enumerate}


\textbf{\large Schlüsselfragen}

\textbf{Frage 1 (Organisation/Durchführung):} Wenn Sie sich etwas hilfreiches begleitend zur Veranstaltung wünschen könnten, was würde dies sein? Explizit Organisatorisch

\textbf{Frage 2 (Durchführung):} Was für Feedback ist während einer
Veranstaltung wichtig? (Worauf kann realistisch noch eingegangen werden?)
\begin{itemize}[noitemsep,topsep=0pt]
    \item Welche Daten werden benötigt in der Nachbereitung?
\end{itemize}

\textbf{Frage 3 (Durchführung/Nachbereitung):} Wenn die Möglichkeit bestände
Teilnehmer während einer Veranstaltung direkt / jeder Zeit anzusprechen, wofür
würden Sie das nutzen?
\begin{itemize}[noitemsep,topsep=0pt]
    \item Wofür nach einer Veranstaltung?
\end{itemize}

\textbf{\large Abschluss}

\begin{enumerate}[noitemsep,topsep=0pt]
    \item Nochmals für die Zeit Danken
    \item Kontaktmöglichkeit im Nachhinein
    \item Verabschiedung
\end{enumerate}

\chapter{Evaluationsfragebogen} \label{appendix:evaluation}


\chapter{Digitale Medien}

Auf der beigefügten CD sind folgende Inhalte zu finden:
\begin{enumerate}[noitemsep,topsep=0pt]
    \item Aufnahmen der geführten Interviews (Verzeichnis \textit{/befragung})
    \item Detaillierte Ergebnisse der Evaluation (Verzeichnis \textit{/evaluation})
    \item Quellcode des entwickelten Systems (Verzeichnis \textit{/framework})
    \item PDF-Version dieser Arbeit
\end{enumerate}



\end{document}
